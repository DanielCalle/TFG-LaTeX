% +--------------------------------------------------------------------+
% | Copyright Page
% +--------------------------------------------------------------------+

\newpage

\thispagestyle{empty}

\begin{center}

{\bf \Huge Resumen}

  \end{center}
\vspace{1cm}

Actualmente, en el mundo cada vez más tecnológico, está surgiendo más y más aplicaciones que satisfacen
a las necesitades de la sociedad, ya sea en el ámbito de la salud, estudio, ocio entre otros.

Nosotros, como amantes de las películas, hemos visto la necesidad de una aplicación de gestión de ocio,
principalmente centrada en el ámbito de las películas. 

Surge de la necesidad de aquellas personas de querer ver una película acompañados, pues es bastante 
incómodo ir al cine solo. Se crean planes con amigos o personas que tengan un gusto similar, pues los mismos
gustos y afinidades conlleva a la elección de una película aceptada por los integrantes. Además, existe la posibilidad
de recomendar a los usuarios cuando estos no saben qué películas ver, básandonos en las elecciones que tomaron
los usuarios que tienen los mismos gustos. Por último, la aplicación ha de tener todas las funciones de gestión de 
datos (usuarios, películas, planes ...) funcionando de forma correcta y fácil de usar.

Para ello, para responder a esta necesidad hemos desarrollado una aplicación móvil 
con la que poder reconocer carteles de películas, incluso permite reconocimiento facial de usuarios, para 
ver información relacionada en tiempo real y poder interactuar con ella a 
través de la Realidad Aumentada (RA). Podremos crear planes para ir a ver la película con otros 
usuarios de la aplicación y además recomendará planes con una película afín a sus gustos.

\vspace{1cm}

% +--------------------------------------------------------------------+
% | On the line below, repla	ce Fecha
% |
% +--------------------------------------------------------------------+

\begin{center}

{\bf \Large Palabras clave}

   \end{center}

   \vspace{0.5cm}
   
   Lista de palabras clave
   \begin{itemize}  
    \item Realidad Aumentada
    \item Sistemas de Recomendación
    \item Filtrado colaborativo
    \item Ocio
    \item Película
    \item Inteligencia artificial
  \end{itemize}
   


