% +--------------------------------------------------------------------+
% | Copyright Page
% +--------------------------------------------------------------------+

\newpage

\thispagestyle{empty}

\begin{center}

{\bf \Huge Resumen}

  \end{center}
\vspace{1cm}

Actualmente, en un mundo cada vez más tecnológico, es muy común que surjan aplicaciones diseñadas
para satisfacer las necesidades de la sociedad en distintos ámbitos, como pueden ser el de la salud, la educación o el ocio entre otros.

Nosotros con nuestro proyecto decidimos centrarnos en el ámbito del ocio y, más concretamente, en el mundo del cine.

Nuestra aplicación surge de la necesidad de aquellas personas que quieren ver una película acompañados ya que, pocas personas hoy en día deciden ir al cine solas y 
podemos estar de acuerdo en que la mayoría de personas prefieren ver una película en compañía.
Para dar respuesta a ésto, nuestra aplicación permite crear planes para ver una cierta película con amigos o personas que tengan un gusto similar, no tendría sentido crear un grupo para ver una película que no le gustara a todos.
Además, existe la posibilidad de que un usuario no sepa que película quiere ver, por lo que la aplicación tendrá la capacidad
de recomendar a los usuarios películas que se ajusten a los gustos que tienen. Por último, para que todo esto funcione, la aplicación tendrá que ser capaz de 
gestionar todos los datos necesarios (usuarios, películas, planes, recomendaciones).

Lo más relevante de nuestra aplicación es la facilidad con la que un usuario puede interactuar con las películas y con las caras de los usuarios. Con esto nos referimos a que, gracias a la realidad aumentada,
podremos reconocer con la cámara de nuestro dispositivo carteles de películas e incluso nos permite el reconocimiento facial de usuarios, para así
obtener explicaciones sobre lo que estamos reconociendo en tiempo real y así poder interactuar a 
través de la realidad aumentada.
\vspace{1cm}

% +--------------------------------------------------------------------+
% | On the line below, repla	ce Fecha
% |
% +--------------------------------------------------------------------+

\begin{center}

{\bf \Large Palabras clave}

   \end{center}

   \vspace{0.5cm}
   
   Lista de palabras clave
   \begin{itemize}  
    \item Realidad Aumentada
    \item Sistemas de Recomendación
    \item Filtrado colaborativo
    \item Ocio
    \item Película
    \item Inteligencia artificial
  \end{itemize}
   


