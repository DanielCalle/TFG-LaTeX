% +--------------------------------------------------------------------+
% | Copyright Page
% +--------------------------------------------------------------------+

\newpage

\thispagestyle{empty}

\begin{center}

{\bf \Huge Resumen}

  \end{center}
\vspace{1cm}
\begin{flushleft}
  Actualmente existen multitud de aplicaciones y herramientas que ayudan a personas con una determinada 
  afición a realizar esta de una manera más sencilla.
  \end{flushleft}
  \begin{flushleft}
  Cuando se planteó la idea de este proyecto observamos encajaba en la definición anterior siendo en este 
  caso en particular el mundo del cine. 
  \end{flushleft}
  \begin{flushleft}
  Surge de la necesidad de aquellas personas que quieren ver una película acompañados, ya sea por amigos o 
  por personas que tengan un gusto similar, siendo la afinidad de las personas a cierta película una característica 
  relevante a la hora de tomar una decisión. Con esto queremos reflejar que cualquier persona que posea un cierto grado 
  de cinefilia podrá tener interés en nuestra idea. Por lo tanto, para responder a esta necesidad hemos desarrollado una 
  aplicación móvil con la que poder reconocer carteles de películas e imágenes de caras de usuarios, para ver información 
  relacionada en tiempo real y poder interactuar con ella a través de la Realidad Aumentada (RA). Podremos crear planes 
  para ir a ver la película con otros usuarios de la aplicación y además recomendará planes con una película afín a sus gustos.
  \end{flushleft}

\vspace{1cm}

% +--------------------------------------------------------------------+
% | On the line below, repla	ce Fecha
% |
% +--------------------------------------------------------------------+

\begin{center}

{\bf \Large Palabras clave}

   \end{center}

   \vspace{0.5cm}
   
   Lista de palabras clave
   \begin{itemize}  
    \item Realidad Aumentada
    \item Sistemas de Recomendación
    \item Unity
    \item Android
    \item Spring
    \item Vuforia
    \item PostgreSQL
    \item Java
    \item C\#
    \item Cloud
    \item Servidor
    \item API Rest
    \item Heroku
  \end{itemize}
   


