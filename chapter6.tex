% +--------------------------------------------------------------------+
% | Sample Chapter 6
% +--------------------------------------------------------------------+

\cleardoublepage

% +--------------------------------------------------------------------+
% | Replace "This is Chapter 6" below with the title of your chapter.
% | LaTeX will automatically number the chapters.
% +--------------------------------------------------------------------+

\chapter{Trabajo futuro}
\label{makereference6}
Uno de los mayores problemas del proyecto fue la dificultad de integrar
 distintas tecnologías entre sí y que estas cumplieran con todos los requisitos
 que buscábamos. Esto ocasiono que el tiempo de aprendizaje y adaptación del uso
 de estas tecnologías se disparará, por tanto, a las primeras funcionalidades que
 teníamos en mente tuvimos que priorizar las que daban el uso básico al problema
 que planteábamos.
A continuación, destacaremos todas esas funcionalidades o mejoras que nos
 gustaría que tuviera en un futuro la aplicación para que la usabilidad y
 posibilidades sean más ricas para el usuario:
\begin{enumerate}
    \item Una de las situaciones que se pueden encontrar nuestros usuarios es
     que quieren ir al cine a ver una película con amigos, pero no les importe
     que película si no pasar un buen rato con buena compañía. En este caso la
     aplicación podría crear planes y una vez se unan todos los amigos sea capaz
     de recomendar la película que mejor se ajusta a sus preferencias.
    \item Otra situación similar a la anterior es que queramos que nos
     recomiende la película que mejor se adapta a los gustos del grupo, pero
     acotando a las películas que elijamos previamente.
    \item En el momento del desarrollo de este proyecto el reconocimiento facial
     no está integrado en las librerías de RA que hemos investigado y la
     complejidad de integrar tecnologías de reconocimiento facial, así como su
     aprendizaje no era viable para el tiempo de este proyecto.
    Con lo que se decidió realizar el reconocimiento del usuario con imágenes
     asemejando esta funcionalidad. Creemos interesante el uso de
     reconocimiento facial para funcionalidades de usuario, aunque pueda ser
     polémica es algo nuevo que aporta valor a la aplicación.
    \item La escena de RA esta implementada en Unity por lo que podemos
     exportarla a otras plataformas como iOS, consiguiendo con esto que la
     aplicación se abra a más usuarios y estos consigan un mayor crecimiento y
     enriquecimiento de la aplicación.
    \item Facilitar la compra de entradas desde la aplicación para que los
     usuarios puedan realizar la mayoría de las acciones, necesarias para ir al
     cine, redireccionando a la página web de venta de los cines a los que se
     quieren acudir para ver la película.
    \item Publicar la aplicación a una plataforma de distribución digital de
    aplicaciones como Google Play, para que los usuarios tengan acceso a
    instalarla en sus dispositivos.
    \item Automatizar con web scraping la obtención de datos de las fuentes de
     información de películas para que la aplicación siempre tenga la
     información actual.
    \item Utilizar distintos idiomas para el contenido de las películas en la
     base de datos del servidor usando una columna que indique el idioma de la
     información.
    \item Una de las cosas más importantes que falta por realizar es una
     evaluación con usuarios que nos proporcionen retroalimentación de la
     aplicación de la cual obtendremos una visión más amplia sobre si satisface
     la necesidad al problema que tratamos en el proyecto o si hay que realizar
     cambios. 
\end{enumerate}