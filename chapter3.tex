% +--------------------------------------------------------------------+
% | Sample Chapter 3
% +--------------------------------------------------------------------+

\cleardoublepage

% +--------------------------------------------------------------------+
% | Replace "This is Chapter 3" below with the title of your chapter.
% | LaTeX will automatically number the chapters.
% +--------------------------------------------------------------------+

\chapter{Diseño de la aplicación}
\label{makereference3}

\section{Stakeholders}
\label{makereference3.1}

\section{Escenarios}
\label{makereference3.2}

\section{Requisitos funcionales}
\label{makereference3.3}

\section{Interfaz de usuario}
\label{makereference3.4}

\section{Sistema de recomendación}
\label{makereference3.5}

\section{Prototipos}
\label{makereference3.6}

\subsection{ARCore} 
\label{makereference3.6.1} 
 
\subsection{ViroMedia} 
\label{makereference3.6.2} 
 
\subsection{Vuforia + Android} 
\label{makereference3.6.3} 
 
En este prototipo se utilizó la librería de vuforia nativa para android, 
el cual hicimos las pruebas de tecnología para reconocer imágenes tanto en  
local como en la nube y después renderizar objetos y textos sobre ello.  
 
1. La librería de vuforia para android está diseñada a muy baja nivel. 

2. Vuforia para dibujar en 3D usa la librería OpenGL.

3. OpenGL utiliza una serie de espacios donde se va colocando las cosas:

Local space: Es el espacio local de cada objeto.
World space: Es el mundo donde se encuentra los objetos.
View space: El mundo visto desde la perspectiva de la cámara.
Clip space: Se integra con la pantalla del móvil y definiendo los límites 
visibles, se establece unas coordenadas de rango -1.0 - 1.0

Las transformaciones de estos espacios se realiza mediante matrices 4x4 
el cual la primera fila significa el punto x, la segunda el punto y y la 
tercera z, mientras que la última columna significa los desplazamientos 
de los objetos de esos ejes.

\includegraphics[height=2.5in]{figures/space-transformation.png}

\includegraphics[height=2.5in]{figures/teapot.png}

3. En OpenGL es necesario escribir código para las tarjetas gráficas, el lenguaje que se usa es GLSL.
este código de GLSL se escribe en forma de String y se llama a un método 
que proporciona OpenGL.

\includegraphics[height=2.5in]{figures/GLSL.png}

4. Otro aspecto es que OpenGL solo ofrece lo básico, no ofrece métodos para dibujar directamente objetos sino 
que hay que seguir un pipeline de procesos para conseguir dibujar algo.

\includegraphics[height=2.5in]{figures/pipeline.png}

Esto consiste en pasar un array de números (cada tres para definir un punto) 
a las tarjetas gráficas, establecer triángulos entre los puntos (más array de números) 
definir colores a partir de los puntos (más arrays)..., y con el código del shader, ejecutar 
estos datos.

5. Por último como sólo ofrece métodos básicos, no hay métodos de escritura de texto, y la forma que 
encontramos y que funcione fue usar un bitmap con los caracteres.

\includegraphics[height=2.5in]{figures/bitmap-font.png}

Rechazamos esto por un principal motivo, para hacer que funciones hay que codificar a muy bajo nivel y 
nos costaría mucho tiempo y esfuerzo.
 
\subsection{Vuforia + Unity} 
\label{makereference3.6.4}