% +--------------------------------------------------------------------+
% | Sample Chapter 3
% +--------------------------------------------------------------------+

\cleardoublepage

% +--------------------------------------------------------------------+
% | Replace "This is Chapter 3" below with the title of your chapter.
% | LaTeX will automatically number the chapters.
% +--------------------------------------------------------------------+

\chapter{Diseño de la aplicación}
\label{makereference3}
En este capítulo detallaremos cómo hemos diseñado nuestra aplicación, desde un análisis de la competencia realizado para 
observar aplicaciones que fueran similares a la nuestra, pasando por una serie de escenarios en los que podría ser usada la aplicación, 
seguido por los requisitos funcionales extraídos de estos escenarios y describiendo finalmente las interfaces de usuario que hemos implementado.

\section{Análisis de la competencia}
\label{makereference3.1}
\begin{flushleft}
    Consideramos que la aplicación que hemos diseñado consta de dos partes claramente diferenciadas (la parte de realidad aumentada
    y la parte de recomendación y gestión de películas), además, pensamos que existen distintos tipos de competencia dependiendo de cada una de ellas.
     \begin{itemize}  
         \item En la parte de recomendación y gestión de películas, compite claramente con aplicaciones como IMDB\cite{imdb} o MovieBase\cite{moviebase}. 
         Estas aplicaciones proporcionan, herramientas para guardar películas y recomendar éstas a los usuarios en función de sus gustos. Estas funcionalidades son 
         muy parecidas a las que nosotros hemos decidido proporcionar a nuestros usuarios, con la diferencia que nuestra aplicación, además de las funcionalidades anteriormente 
         mencionadas, permite realizar más acciones sobre las películas como crear planes con amigos.
        \item A diferencia de la parte de recomendación y gestión en donde hay una gran variedad de aplicaciones que ofrecen servicios parecidos a los nuestros, únicamente hemos 
        encontrado una aplicación (Paramount AR+\cite{paramountar}) que ofrezca servicios parecidos a los nuestros relacionados con Realidad Aumentada. Según las especificaciones de la aplicación permite identificar pósteres y 
        mostrar información sobre la película detectada, servicio muy parecido al que nosotros hemos proporcionado. 
    \end{itemize}
    Para intentar alejarnos de esta competencia, hemos decidido unir ambas funcionalidades en una aplicación y, además, proporcionar al usuario más funcionalidades como escanear una aplicación con Realidad Aumentada, 
    ver su información y añadirla a un plan con amigos para ver dicha película. Para funcionalidades como ésta y otras en las que unimos la realidad aumentada con la recomendación de películas y gestión de planes no hemos encontrado actualmente ninguna aplicación que proporcione estos servicios.
    \\
    Además, la gestión de amigos con Realidad Aumentada es una funcionalidad que, tras nuestro análisis de la competencia, solamente la posee nuestra aplicación. 
    Se basa en enfocar con la cámara a la imagen de un usuario de la aplicación y que mediante realidad aumentada muestre visualmente la información de los tres 
    planes activos que tiene este usuario, una vez ha sido añadido como amigo, que más podrían gustarme y cuanto se estima que me gustaría ese plan.
    
\end{flushleft}
\section{Escenarios}
\label{makereference3.2}
Antes de decidir como implementaríamos nuestra aplicación decidimos que sería importante establecer una serie
de escenarios en los que podría ser usada. Tras esto pudimos ponernos de acuerdo en qué funcionalidades incluiríamos y que 
éstas se ajustasen a las necesidades de los usuarios. Además, estos escenarios deberían usar la Realidad Aumentada para que 
les aporte valor, ya que, como hemos expuesto anteriormente en el capítulo 2, el grueso de nuestra aplicación se centra en esta tecnología.

\paragraph{Escenario 1. Reconociendo carteles de películas.}
\begin{flushleft}
    Te apetece ir al cine y encuentras el cartel de una película que te interesa ver en una revista. Sacas tu móvil, abres la aplicación y 
    la escaneas. Con la realidad aumentada serás capaz de ver la valoración de la película, acceder a su tráiler en Youtube, ver la información
    referente a la película (título, director, duración, sinopsis) y guardarla como favorita.
    
\end{flushleft}
\paragraph{Escenario 2. Reconociendo usuarios.}
\begin{flushleft}
    Encuentras a otra persona que también usa la aplicación, por lo que te interesaría mantener contacto con ella para saber si tenéis los mismos 
    gustos. La aplicación te permitirá reconocer una imagen de su cara para poder añadirle como amigo mediante la realidad aumentada. 
\end{flushleft}
\paragraph{Escenario 3. Obtener recomendación.}
\begin{flushleft}
    Una vez seáis amigos en la aplicación,
    la próxima vez que se reconozca la imagen de la cara del usuario aparecerán planes con películas que os puedan interesar a los dos, permitiéndote unirte o decidir cuál es el
    que más se ajusta a vuestros gustos. 
\end{flushleft}


\paragraph{Escenario 4. Quedar con amigos para ver una película.}
\begin{flushleft}
    Una vez que has guardado una película porque te interesa ir a verla con alguien, ya sea que quieras verla en el cine o en tu casa, sea fin de semana o no, la aplicación te facilitará quedar 
    con las personas que estén interesadas en verla, además de que la aplicación supondrá un apoyo para que los usuarios tomen sus decisiones, mostrándoles cuánta afinidad a dicha película tienen.
\end{flushleft}

\paragraph{Escenario 5. Usar la aplicación sin la realidad aumentada.}
\begin{flushleft}
     Si no te apetece usar la cámara de tu dispositivo también podrás acceder a toda la funcionalidad descrita anteriormente solo con el uso de la aplicación, podrás buscar planes, crearlos, borrarlos y unirte a ellos, también podrás buscar películas y valorarlas y buscar usuarios añadirlos. También se te 
     recomendarán películas basadas en tus gustos.
\end{flushleft}

\section{Requisitos funcionales}
\label{makereference3.3}

\paragraph{\large Planes:\\}

Cuando hablamos de un plan, nos referimos a un elemento creado a partir de una película, un plan para un usuario significa que ese usuario tiene interés en ver la película con la que se creó dicho plan.
Los planes están compuestos por una película, un usuario creador del plan, y los usuarios que se hayan unido.
\\
\textbf{Funciones:}
\begin{enumerate}
    \item Crear plan: un usuario crea un plan a partir de una película que ha guardado, con una descripción, fecha, hora y lugar determinados.
    \item Unirse a un plan: solo podrán unirse los amigos del usuario creador del plan.
    \item Salirse de un plan: una vez te has unido al plan puedes salir del mismo.
    \item Eliminar un plan: solo puede eliminar el plan la persona que lo ha creado.
    \item Información de un plan: la película que se quiere ver, los usuarios que se han unido, el usuario que lo ha creado, la fecha, el lugar, una descripción y la hora.
\end{enumerate} 
\paragraph{\large Películas guardadas:\\}

Las películas guardadas son aquellas que el usuario ha decidido guardar para ver su información más tarde o para posteriormente añadirlas a un plan.
\\
\textbf{Funciones:}
\begin{enumerate}
    \item Guardar película: cuando un usuario reconoce una película con la cámara, se le permitirá guardar la película como favorita. También podrá guardar la película si la busca desde la aplicación sin necesidad de la realidad aumentada.
    \item Información de una película: un usuario puede ver la información de una película que haya guardado. Podrá ver una pequeña sinopsis, el género, el director, su valoración y su tráiler en Youtube.
    \item Valorar una película: también se le permite a un usuario valorar una película a su gusto.
\end{enumerate} 
\paragraph{\large Usuarios:\\}
Usuario es toda aquella persona que se haya registrado en la aplicación y se haya creado un perfil.
\\
\textbf{Funciones:}
\begin{enumerate}
    \item Añadir como amigo: cuando un usuario reconoce la imagen de otro se le permitirá añadirle como amigo. También podrá hacerlo si le busca desde la aplicación sin necesidad de la realidad aumentada.
    \item Eliminar un amigo: existe la posibilidad de eliminar a un usuario de tu lista de amigos.
    \item Información de un usuario: un usuario puede ver la información de otro usuario, como su nombre y su foto de perfil. Si son amigos podrá ver además 3 planes que le interesan a ambos.
\end{enumerate} 

\section{Interfaz de usuario}
\label{makereference3.4}

Hemos dividido nuestra aplicación en 3 contenedores diferentes: la interfaz de mis planes, la de recomendaciones y la de mis películas. Además, poseemos una serie de interfaces de realidad aumentada que mostraremos. A continuación hablaremos
de las interfaces de usuario más relevantes que hemos diseñado:
\subsection{Interfaz principal}
\label{makereference3.4.1}
Tenemos una barra en la parte inferior para movernos por las 3 secciones de nuestra aplicación: mis planes, recomendaciones y mis películas.
Cada sección incluye un icono para apoyar a la representación.
Además, encontramos un botón con un símbolo de una cámara en todas las secciones que nos permitirá acceder a la interfaz de realidad aumentada.
\subsection{Interfaz de mis planes}
\label{makereference3.4.2}
El objetivo de esta vista es el de mostrar los planes públicos que existen, es decir, aquellos en los que estás o hayas creado, indicando con una imagen de fondo la película para la que se creó el plan, el usuario
que creó dicho plan y los usuarios que se han unido. 
Podemos observar como cada elemento representa un plan, con el fondo siendo el cartel de la película, la fecha en la que tendrá lugar el plan en la esquina superior izquierda,
justo debajo el título del plan, arriba a la derecha el número del plan (algo simple pero que nos indica
claramente lo que representa este elemento) y en la parte inferior aparecen las fotos de los usuarios que se han unido al plan.
Todo esto podemos verlo en la Figura 3.1.
\begin{figure}[H]
    \centering
    \includegraphics[height=4in]{figures/plansList.jpg}
    \caption{Lista de planes}
    \label{fig:birds}
\end{figure}

\subsection{Interfaz de información de un plan}
\label{makereference3.4.3}
Esta interfaz aparece cuando pulsamos en un plan, ver Figura 3.2. Nos aparecerá en grande la imagen de la película que se quiere ir a ver con dicho plan, junto con información
relevante de dicho plan, que podemos ver en la Figura 3.3:
\begin{itemize}
    \item Fecha: día, mes y año en el que tendrá lugar dicho plan.
    \item Hora: hora del plan.
    \item Lugar: localización que haya puesto el creador del plan para ver la película.
    \item Descripción: breves anotaciones características que haya escrito el creador sobre el plan.
    \item Usuarios unidos: imágenes de los usuarios que se han unido al plan.
\end{itemize}
La información de cada plan la introduce el usuario mediante un formulario muy simple cuando crea un plan con una película.
Además, nos aparece un botón en la parte inferior con el texto: UNIRSE AL PLAN o SALIR DEL PLAN, según estemos o no ya dentro del plan.
También en la parte inferior derecha de la imagen de la película tenemos un botón que nos redirige a la interfaz de información de dicha película.
En la parte superior derecha nos aparecería un icono de una basura, lo que nos permitirá borrar el plan, si somos el creador.
\begin{figure}[H]
    \centering
    \makebox[0pt][c]{%
    \begin{minipage}[b]{0.5\linewidth}
    \centering
      \includegraphics[height=4in]{figures/infoPlan1.jpg}
      \caption{Información del plan}
    \label{sva}
    \end{minipage}%
    \hspace{0.2cm}
    \begin{minipage}[b]{0.5\linewidth}
    \centering
     \includegraphics[height=4in]{figures/infoPlan2.jpg}
      \caption{Información del plan}
    \label{svb}
    \end{minipage}%
    }%
\end{figure}
\subsection{Interfaz de mis películas}
\label{makereference3.4.4}
Para mostrar las películas guardadas decidimos usar una cuadrícula que muestre solamente los carteles de las películas, es una interfaz simple pero muy visual. Para acceder a la información
de cada película simplemente debemos pulsar en uno de los carteles.
\begin{figure}[H]
    \centering
    \includegraphics[height=4in]{figures/FilmsList.jpg}
    \caption{Lista de películas guardadas}
\end{figure}
En la figura 3.4 podemos observar cómo es la interfaz. En este caso el usuario solo ha guardado 2 películas.
\subsection{Interfaz de información de una película}
\label{makereference3.4.5}

Tras pulsar en una de las películas que hemos guardado nos aparecerá dicha interfaz. 
Esta vista nos permite ver en primer plano el cartel de la película y la información correspondiente a la misma, como puede
ser la sinopsis de la película, el género y el nombre del director, podemos observarlo en la Figura 3.5 y la Figura 3.6. Además, poseemos un indicador de calibre circular que nos permite, al pulsarlo,
valorar la película mediante un deslizador.
Si hacemos scroll hacia abajo, la imagen de la película se irá ocultando para ofrecernos una mejor visión de la información de la película.
En la parte inferior se nos presentan dos botones, uno para crear un plan con la película con el texto: AÑADIR AL PLAN. El otro botón con el símbolo de reproducir un vídeo, nos llevará al tráiler de la película en Youtube.
Arriba a la derecha observamos el icono de un corazón, lo que nos permitirá quitar esta película de nuestras favoritas y ya no aparecerá en la
lista de películas guardadas.

\begin{figure}[H]
    \centering
    \makebox[0pt][c]{%
    \begin{minipage}[b]{0.5\linewidth}
    \centering
      \includegraphics[height=4in]{figures/infoFilm1.jpg}
      \caption{Información de la película}
    \label{sva}
    \end{minipage}%
    \hspace{0.2cm}
    \begin{minipage}[b]{0.5\linewidth}
    \centering
     \includegraphics[height=4in]{figures/infoFilm2.jpg}
      \caption{Información de la película}
    \label{svb}
    \end{minipage}%
    }%
\end{figure}
\subsection{Interfaz de recomendaciones}
\label{makereference3.4.6}
En esta interfaz tenemos presentes 3 secciones de recomendaciones, cada una con carteles de películas junto al título debajo que se le recomiendan al usuario. Cada sección presenta scroll horizontal para poder
visualizar todas las películas que se le recomiendan en dicha sección, ver en Figura 3.7 y 3.8. La primera sección son las películas recomendadas al usuario según sus gustos, la segunda representa las películas más populares
y las películas que están siendo estrenadas. Al pulsar en cada una de ellas accederemos a la información de cada película (Figura 3.5) pudiendo guardarla como favorita si no lo hemos hecho ya.
\begin{figure}[H]
    \centering
    \makebox[0pt][c]{%
    \begin{minipage}[b]{0.5\linewidth}
    \centering
      \includegraphics[height=4in]{figures/recommendations1.jpg}
      \caption{Recomendaciones de películas}
    \label{sva}
    \end{minipage}%
    \hspace{0.2cm}
    \begin{minipage}[b]{0.5\linewidth}
    \centering
     \includegraphics[height=4in]{figures/recommendations2.jpg}
      \caption{Recomendaciones de películas}
    \label{svb}
    \end{minipage}%
    }%
\end{figure}

\subsection{Interfaz de realidad aumentada principal}
\label{makereference3.4.4}
Una interfaz muy simple en la que tenemos una línea horizontal que se mueve de arriba hacia abajo y viceversa en la pantalla como si se tratara de un escáner, para que el usuario
pueda entender que debe escanear una imagen, como podemos ver en la Figura 3.9.
\begin{figure}[H]
    \centering
    \includegraphics[height=4in]{figures/escaner.jpg}
    \caption{Escáner}
\end{figure}
\subsection{Interfaz de realidad aumentada tras reconocer un cartel de película}
\label{makereference3.4.4}
En esta interfaz destacamos los distintos componentes que aparecen al usar la cámara de nuestro teléfono y reconocer el cartel de una película.
Cuando reconocemos un cartel, sobre la imagen de la película nos aparecerá en la parte superior una valoración numérica de la película, la cual la hemos 
sacado de IMDB. justo en el medio de la imagen encontramos el icono típico de Youtube que, al pulsarlo, nos abrirá nuestra aplicación de Youtube para que
el usuario pueda ver el tráiler de esta película. En la esquina inferior derecha encontramos otro icono que al ser pulsado lo que hará será desplegar una serie de botones con distinta funcionalidad,
uno de ellos nos redirige a la información de la película en IMDB, que sería el icono representado por una letra i, el otro botón representado por una mano con un pulgar hacia arriba nos permitirá
guardar la película en nuestra lista de películas guardadas, para posteriormente acceder a su información, valorarla o crear un plan con ella. Todo esto lo podemos observar en la figura 3.10.

\begin{figure}[H]
    \centering
    \includegraphics[height=4in]{figures/filmrecognized2.jpg}
    \caption{Realidad aumentada tras reconocer una película}
\end{figure}

\subsection{Interfaz de realidad aumentada al reconocer a un usuario que no es amigo}
\label{makereference3.4.5.1}
\begin{flushleft}
Se trata de una interfaz sencilla en donde a la persona que está usando la cámara se le proporciona el nombre del usuario al
 que está enfocando (siempre que el usuario enfocado esté registrado) junto a un botón de añadir amigo, como podemos ver en la Figura 3.11.
\end{flushleft}
\begin{figure}[H]
    \centering
    \includegraphics[height=4in]{figures/usernotFriendrecognized.jpg}
    \caption{Realidad aumentada tras reconocer a un usuario que no es amigo}
\end{figure}
\subsection{Interfaz de realidad aumentada al reconocer a un usuario que sí es amigo}
\label{makereference3.4.5.2}
\begin{flushleft}
Esta interfaz aparece al enfocar con la cámara a un usuario que es amigo del que enfoca, 
es una interfaz más compleja que la anteriormente descrita.
\end{flushleft}
\begin{flushleft}
Cuando un usuario enfoca a otro que es amigo, se proporciona la opción de incorporarse a uno de los planes del
 amigo. Para ello, se le muestra una interfaz con los tres planes que, según nuestro algoritmo de recomendación,
  más pueden interesarle de los que su amigo tiene. En caso de que haya menos de tres planes para mostrar, mostrará esos.
\end{flushleft}
\begin{flushleft}
La interfaz proporciona la imagen de las películas de los tres planes, además, mediante una interfaz dinámica 
podemos movernos de un plan a otro, de esta forma podemos ver que usuarios hay en cada plan.
\end{flushleft}
\begin{figure}[H]
        \centering
        \includegraphics[width=3in, angle=270]{figures/chapter-2/CapturaRecomendador.JPG}
        \caption{Vista encargada de recomendar planes en RA al enfocar a un amigo}
\end{figure}
\begin{flushleft}
Como se puede ver en la Figura 3.12, en la interfaz aparecen tres películas, esto se debe a que el recomendador ha encontrado tres planes que al usuario le pueden gustar. Únicamente se muestra la 
información relacionada con los usuarios del plan situado en el centro para no saturar la interfaz. Mediante los 
botones de izquierda y derecha el usuario puede moverse entre planes para poder ver la información necesaria 
de cada plan y elegir aquel que más le guste.
\end{flushleft}
\begin{flushleft}
La información visual que aparece relacionada con los usuarios representa:
\begin{itemize}
    \item Arriba a la izquierda, la foto del usuario que está valorando unirse al plan, es decir, el que ha enfocado a su amigo para ver en cuál de sus planes se puede unir. Además, aparece una representación 
    gráfica de cuanto estima el algoritmo de recomendación que le puede gustar la película.
    \item Arriba a la derecha el amigo que se encuentra ya en el plan y al que se le está apuntando con la cámara.
    \item El resto de las posiciones (hasta 6) identifican al resto de usuarios que se hayan unido a ese plan.
\end{itemize}
Además, aparece un botón situado debajo del plan central que permite al usuario que lo pulse unirse al 
plan. Este botón mostrará una interfaz como la de la Figura 3.2 y 3.3 para que el usuario pueda ver todos 
los detalles del plan y decidir si se quiere unir.
\end{flushleft}

\begin{flushleft}
     En este capítulo hemos observado los distintos escenarios que hemos planteado para el uso de nuestra aplicación, a partir de ellos hemos sacado una serie de 
     requisitos funcionales que posteriormente han sido mejor especificados tras las descripciones de las interfaces de usuario que hemos diseñado, apoyándonos en imágenes 
     para mostrar visualmente a lo que nos referimos en cada subsección de las interfaces de usuario.
\end{flushleft}





