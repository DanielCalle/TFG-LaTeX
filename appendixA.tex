% +--------------------------------------------------------------------+
% | Appendix A Page (Optional)                                         |
% +--------------------------------------------------------------------+

\cleardoublepage

\chapter{Guía de instalación para el servidor}
\label{app:guia_instalacion}

\section{Docker}
\label{app:docker}
Esta es una guía para el despliegue del servidor en Docker.

El repositorio del proyecto del servidor se encuentra en: 

\url{https://github.com/DanielCalle/TFG-Server}

Para ello primero tenemos que tener docker instalado, el cual 
se puede instalar por el siguiente enlace:

\url{https://www.docker.com/}

Tras descargarse el repositorio e iniciado docker, situarse en la carpeta raíz del proyecto mediante el comando \textbf{cd}.

\begin{lstlisting}[language=bash, caption=Despliegue de la instancia de docker]
    # Ir a la raiz del proyecto
    cd <path-del-proyecto> 
    # Despliegue
    docker-compose up
\end{lstlisting}

Con la sentecia \textbf{docker-compose up} se despliega la instancia de docker con las configuraciones 
que están en el archivo docker-compose.yml. Esperar hasta que en el terminal de comandos aparezcan las líneas
de la \autoref{fig:docker}

\begin{figure}[H]
    \centering
    \includegraphics[width=4in]{figures/appendix-A/docker.png}
    \caption{Docker desplegado}
    \label{fig:docker}
\end{figure}

Ahora que se ha levantado la instancia de docker y sus respectivos puertos mapeados.
Abrir un nuevo terminal de comandos y conectarse al contenedor mediante ssh con la contraseña \textbf{tfg-ucm}

\begin{lstlisting}[language=bash, caption=Conexión ssh]
    # El puerto 22022 es el que se habia mapeado
    ssh root@localhost -p 22022
\end{lstlisting}

Una vez conectado al contenedor, hay que ir a la ruta en la que se encuentra el proyecto: \url{/usr/src/app}

\begin{lstlisting}[language=bash, caption=Ir a la ruta del proyecto]
    # Ir a ruta
    cd /usr/src/app
\end{lstlisting}

Antes de levantar el servicio, hay que preparar postgresql. 
Hay que ejecutar el script de configuración llamado \textbf{database.sh}. Para ello hace falta primero darle permiso de ejecución al 
script.

\begin{lstlisting}[language=bash, caption=Configuración postgresql]
    # Dar permiso de ejecucion al script
    chmod +x database.sh
    # Ejecutar el script
    ./database.sh
\end{lstlisting}

Una vez configurado postgres, se procede al despliegue del servicio. Con \textbf{service.sh}.

\begin{lstlisting}[language=bash, caption=Despliegue]
    # Dar permiso de ejecucion al script
    chmod +x service.sh
    # Para iniciar el servicio
    ./service.sh start
    # Para reiniciar el servicio
    ./service.sh restart
    # Para parar el servicio
    ./service.sh stop
\end{lstlisting}

Puede que haya problemas con el formato 
de los scripts pues los scripts se escribieron en Windows, 
para ello hace falta descargarse \textbf{vim}, y cambiarle el formato a los scripts.

\begin{lstlisting}[language=bash, caption=Ayudas]
    # Descarga de vim
    apt-get install -y vim
    # Cambiar formato
    vim script.sh
    :set fileformat=unix
    :wq
\end{lstlisting}

Una vez terminada la ejecución, ya está levantado el servicio en el puerto 8080, 
se puede probar con el siguiente enlace en cualquier navegador
\url{localhost:8080/films}.

Si se quiere ver la estructura de la base de datos se encuentra en 
el siguiente fichero: \url{path-al-proyecto/src/main/webapp/WEB-INF/sql/filmar_low.sql}.

Cuando queramos terminar el proyecto, habríamos que desconectar la conexión de ssh y apagar la instancia de docker.

\begin{lstlisting}[language=bash, caption=Salir de ssh y docker]
    # Para desconectar la conexion con el ssh
    exit 
    # Para apagar la instancia de docker
    docker-compose down
\end{lstlisting}

\section{Heroku}
\label{app:heroku}
Para comenzar debemos crear una cuenta en \href{https://www.heroku.com/}{Heroku}.
Una vez la tenemos creamos una nueva aplicación a la
 cual daremos un nombre como vemos en la \autoref{fig:heroku_1} y este
 nombre a su vez será parte de la URL pública.
\begin{figure}[H]
    \centering
    \includegraphics[width=6in]{figures/appendix-A/heroku_1.png}
    \caption{Crear nueva aplicación en Heroku}
    \label{fig:heroku_1}
\end{figure}
Cuando creemos la aplicación accederemos a la configuración en la que veremos
 opciones como en la figura \autoref{fig:heroku_2}.
\begin{figure}[H]
    \centering
    \includegraphics[width=6in]{figures/appendix-A/heroku_2.png}
    \caption{Configuración de la aplicación}
    \label{fig:heroku_2}
\end{figure}
Entramos en la pestaña de recursos y buscamos Postgres como vemos en la
\autoref{fig:heroku_3}. La seleccionamos y agregamos la versión gratuita.
\begin{figure}[H]
    \centering
    \includegraphics[width=6in]{figures/appendix-A/heroku_3.png}
    \caption{Añadir nuevos recursos}
    \label{fig:heroku_3}
\end{figure}
Si accedemos al recurso de la base de datos que acabamos de crear y Entramos
 en la configuración podemos ver las credenciales que necesitamos añadir a
 nuestro servidor para que funcione correctamente como vemos en la
 \autoref{fig:heroku_4}.
\begin{figure}[H]
    \centering
    \includegraphics[width=6in]{figures/appendix-A/heroku_4.png}
    \caption{Ver credenciales}
    \label{fig:heroku_4}
\end{figure}
Si vamos a la pestaña de despliegue de la aplicación, como vemos en la
 \autoref{fig:heroku_5} nos ofrece diversos métodos de
 despliegue:
\begin{itemize}
    \item Heroku Git
    \item GitHub
    \item Container Registry
\end{itemize}
\begin{figure}[H]
    \centering
    \includegraphics[width=6in]{figures/appendix-A/heroku_5.png}
    \caption{Métodos de despliegue}
    \label{fig:heroku_5}
\end{figure}
Elegimos la opción de GitHub que nos proporciona un despliegue fácil y rápido.
Tenemos que vincular la cuenta de nuestro GitHub (Importante ser el dueño del
 repositorio).
\begin{figure}[H]
    \centering
    \includegraphics[width=6in]{figures/appendix-A/heroku_6.png}
    \caption{Opciones de despliegue utilizando Github}
    \label{fig:heroku_6}
\end{figure}

Una vez vinculada la cuenta de GitHub tenemos dos métodos de despliegue como
 vemos en la \autoref{fig:heroku_6}. Un método automático para desplegar los
 cambios de una rama del repositorio que se accionará cada vez que subamos
 cambios y otro método manual que se desplegará sólo cuando presionemos
 el botón. 
Cuando elijamos una de las opciones, la aplicación estará lista para usarse.

\section{Guía para generar un apk}
\label{app:apk}
Como hemos comentado anteriormente, nuestra aplicación consta de dos partes: La parte de Realidad Aumentada y 
la parte de Android. Por ello, a la hora de generar una apk completa de la aplicación habrá que generar estas partes 
por separado.

\subsection{Generar la parte de la aplicación en Unity.}
Para generar la aplicación en Unity:
\begin{enumerate}
    \item Abrimos en Unity el proyecto. El nombre de este es TFG-Vuforia.
    \item A continuación debemos seleccionar la escena en la que se encuentra la aplicación en Unity. Esta escena se encuentra en 
    el directorio principal Assets > Scenes y su nombre es CloudRecognition.
    \begin{figure}[H]
        \centering
        \includegraphics[width=6in]{figures/Appendix-A/CapturaCloudRecognition.JPG}
        \caption{Ubicación de la escena principal}
        \label{fig:CloudRecognitionUbication1}
    \end{figure}
    \item Una vez localizada la escena principal, debemos seleccionarla. Para ello la arrastraremos al panel de trabajo (situado 
    arriba a la izquierda).
    \begin{figure}[H]
        \centering
        \includegraphics[width=4in]{figures/Appendix-A/CapturaPanel.JPG}
        \caption{Panel principal}
        \label{fig:CloudRecognitionUbication2}
    \end{figure}
    \item El siguiente paso es exportar este proyecto en Unity para que pueda usarlo la parte de Android. Para ello, pulsaremos 
    File > Build Settings. Aparecerá una vista como la de la Figura \ref{fig:BuildSettings}, es muy importante antes de exportar la 
    aplicación a Android tener seleccionado Android (a la izquierda). Para seleccionarlo se marca y se presiona el botón de Switch Platform.

    Una vez que nos hemos asegurado de tener seleccionada la plataforma Android ya podemos exportar el proyecto. Para ello seleccionamos la opción 
    que dice Export proyect con el Build system de Gradle, como se indica en la Figura \ref{fig:BuildSettings}.
    \begin{figure}[H]
        \centering
        \includegraphics[width=4in]{figures/Appendix-A/CapturaBuildSettings.JPG}
        \caption{Build Settings}
        \label{fig:BuildSettings}
    \end{figure}
\end{enumerate}
\subsection{Cómo unir la parte de Unity y Android}
Como solución de la sección anterior, se habrá generado en la carpeta especificada un proyecto en Android llamado FilmAR.

Si fuésemos a crear un proyecto en Android desde cero, partiríamos desde este proyecto generado por Unity. 

Si lo que queremos es actualizar nuevos cambios en Unity en la aplicación de Android seguiremos los siguientes pasos:
\begin{enumerate}
    \item Accederemos a los directorios FilmAR > src > main de las dos aplicaciones (la generada por Unity con los nuevos cambios 
    y la aplicación en Android general de la aplicación).
    \item Eliminaremos la carpeta Assets de la aplicación general y la cambiaremos por la nueva carpeta Assets generada por Unity.
\end{enumerate}
De esta manera, tendremos la aplicación en Android actualizada a los nuevos cambios de la parte de Realidad Aumentada.
\subsection{Generar un APK}
Para generar un APK, simplemente tendremos que ir en Android Studio a la pestaña Build y hacer click en generar APK. De esta forma podremos 
descargar la aplicación.