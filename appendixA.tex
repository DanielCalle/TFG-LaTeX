% +--------------------------------------------------------------------+
% | Appendix A Page (Optional)                                         |
% +--------------------------------------------------------------------+

\cleardoublepage

\chapter{Guía de instalación para el servidor}
\label{app:guia_instalacion}

\section{Docker}
\label{app:docker}
Esta es una guía para el despliegue del servidor en Docker.

El repositorio del proyecto del servidor se encuentra en:
\url{https://github.com/ZReKoJ/Docker.git}

Para ello primero tenemos que tener docker instalado, el cual 
se puede instalar por el siguiente enlace \url{https://www.docker.com/}

Tras descargarse el repositorio, mediante commando, situarse en la carpeta raíz del proyecto.

\begin{lstlisting}[language=bash, caption=Despliegue de la instancia de docker]
    # Despliegue
    docker-compose up
\end{lstlisting}

Con la sentecia \textbf{docker-compose up} se despliega la instancia de docker con las configuraciones 
que están en el archivo docker-compose.yml.

Ahora que se ha levantado la instancia de docker y sus respectivos puertos mapeados.
Se conecta al container mediante ssh con la contraseña \textbf{tfg-ucm}

\begin{lstlisting}[language=bash, caption=Conexión ssh]
    # El puerto 22022 es el que se habia mapeado
    ssh root@localhost -p 22022
\end{lstlisting}

Una vez conectado al container, hay que ir a la ruta en el que se encuentra el proyecto: \url{/usr/src/app}

Antes de levantar el servicio, hay que preparar postgresql. El archivo SQL está dentro del proyecto
en el siguiente path: \url{/src/main/webapp/WEB-INF/sql/filmar_low.sql}. En la raíz del proyecto se 
encuentra un script llamado \textbf{database.sh}.

\begin{lstlisting}[language=bash, caption=Configuración postgresql]
    ./database.sh
\end{lstlisting}

Una vez configurado postgres, se procede al despliegue del servicio. Con \textbf{service.sh}.

\begin{lstlisting}[language=bash, caption=Despliegue]
    # Para parar el servicio
    ./service.sh start
    # Para reiniciar el servicio
    ./service.sh restart
    # Para parar el servicio
    ./service.sh
\end{lstlisting}

Para ejecutar los scripts hacen falta añadirles permiso de ejecución. Puede que halla problemas con el formato 
de los scripts pues los scripts se escribieron en Windows, para ello hace falta descargarse \textbf{vim}, y cambiarle el formato a los scripts.

\begin{lstlisting}[language=bash, caption=Ayudas]
    # Dar permiso de de ejecucion
    chmod +x script.sh
    # Descarga de vim
    apt-get install -y vim
    # Cambiar formato
    vim script.sh
    :set fileformat=unix
    :wq
\end{lstlisting}

Ya está levantado el servicio en el puerto 8080, se puede probar con el siguiente enlace
\url{localhost:8080/films}.

\section{Heroku}
\label{app:heroku}
Para comenzar debemos crear una cuenta en \href{https://www.heroku.com/}{Heroku}.
Una vez la tenemos creamos una nueva aplicación a la
 cual daremos un nombre como vemos en la \autoref{fig:heroku_1} y este
 nombre a su vez será parte de la URL pública.
\begin{figure}[H]
    \centering
    \includegraphics[width=6in]{figures/appendix-A/heroku_1.png}
    \caption{Crear nueva aplicación en Heroku}
    \label{fig:heroku_1}
\end{figure}
Cuando creemos la aplicación accederemos a la configuración en la que veremos
 opciones como en la figura \autoref{fig:heroku_2}.
\begin{figure}[H]
    \centering
    \includegraphics[width=6in]{figures/appendix-A/heroku_2.png}
    \caption{Configuración de la aplicación}
    \label{fig:heroku_2}
\end{figure}
Entramos en la pestaña de recursos y buscamos Postgres como vemos en la
\autoref{fig:heroku_3}. La seleccionamos y agregamos la versión gratuita.
\begin{figure}[H]
    \centering
    \includegraphics[width=6in]{figures/appendix-A/heroku_3.png}
    \caption{Añadir nuevos recursos}
    \label{fig:heroku_3}
\end{figure}
Si accedemos al recurso de la base de datos que acabamos de crear y Entramos
 en la configuración podemos ver las credenciales que necesitamos añadir a
 nuestro servidor para que funcione correctamente como vemos en la
 \autoref{fig:heroku_4}.
\begin{figure}[H]
    \centering
    \includegraphics[width=6in]{figures/appendix-A/heroku_4.png}
    \caption{Ver credenciales}
    \label{fig:heroku_4}
\end{figure}
Si vamos a la pestaña de despliegue de la aplicación, como vemos en la
 \autoref{fig:heroku_5} nos ofrece diversos métodos de
 despliegue:
\begin{itemize}
    \item Heroku Git
    \item GitHub
    \item Container Registry
\end{itemize}
\begin{figure}[H]
    \centering
    \includegraphics[width=6in]{figures/appendix-A/heroku_5.png}
    \caption{Métodos de despliegue}
    \label{fig:heroku_5}
\end{figure}
Elegimos la opción de GitHub que nos proporciona un despliegue fácil y rápido.
Tenemos que vincular la cuenta de nuestro GitHub (Importante ser el dueño del
 repositorio).
\begin{figure}[H]
    \centering
    \includegraphics[width=6in]{figures/appendix-A/heroku_6.png}
    \caption{Opciones de despliegue utilizando Github}
    \label{fig:heroku_6}
\end{figure}
Una vez vinculada la cuenta de GitHub tenemos dos métodos de despliegue como
 vemos en la \autoref{fig:heroku_6}. Un método automático para desplegar los
 cambios de una rama del repositorio que se accionará cada vez que subamos
 cambios y otro método manual que se desplegará sólo cuando presionemos
 el botón. 
Cuando elijamos una de las opciones, la aplicación estará lista para usarse.