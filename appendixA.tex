% +--------------------------------------------------------------------+
% | Appendix A Page (Optional)                                         |
% +--------------------------------------------------------------------+

\cleardoublepage

\chapter{Guía de instalación para el servidor}
\label{app:guia_instalacion}

\section{Docker}
\label{app:docker}
Esta es una guía para el despliegue del servidor en Docker.

Para ello primero tenemos que tener una instancia de docker, si se quiere ejecutar en local, 
hay que instalar docker por el enlace \url{https://www.docker.com/}
La imagen de docker se descarga por git por el siguiente enlace:
\url{https://github.com/ZReKoJ/Docker.git}

\begin{lstlisting}[language=bash, caption=Dockerfile]
    FROM ubuntu:16.04

    RUN apt-get update 
    RUN apt-get install -y openssh-server
    RUN mkdir /var/run/sshd
    RUN echo 'root:tfg-ucm' | chpasswd
    RUN sed -i 's/PermitRootLogin prohibit-password/PermitRootLogin
         yes/' /etc/ssh/sshd_config
    
    # SSH login fix. Otherwise user is kicked off after login
    RUN sed 's@session\s*required\s*pam_loginuid.so@session optional
         pam_loginuid.so@g' -i /etc/pam.d/sshd
    
    ENV NOTVISIBLE "in users profile"
    RUN echo "export VISIBLE=now" >> /etc/profile
    
    EXPOSE 22 8000 8080
    CMD ["/usr/sbin/sshd", "-D"]
\end{lstlisting}

\begin{lstlisting}[language=bash, caption=Descarga de la imagen de docker]
    # Descarga de la imagen
    git clone https://github.com/ZReKoJ/Docker.git 
    # Entrar en la carpeta
    cd Docker 
    # Crear la imagen
    docker build --tag=filmar . 
\end{lstlisting}

\begin{figure}[H]
    \centering
    \includegraphics[width=6in]{figures/appendix-A/list-docker-images.png}
    \caption{Imágenes creadas de docker}
\end{figure}

Se han creado dos imágenes tras la ejecución, la de ubuntu es una dependencia usada, para tener el entorno de Linux, por eso no se puede borrar.

\begin{lstlisting}[language=bash, caption=Levantamiento de la instancia de docker]
    # Levanta docker mapeando puertos
    # Tras la ejecutar se va a quedar corriendo
    # Pulse Ctrl + C para parar
    # Pero estaria levantado el contenedor
    docker run -p 22022:22 -p 8080:8080 filmar 
    # No hace falta ejecutar los siguientes comandos
    # Comando para listar containers de docker corriendo
    docker ps 
    # Listar todos los containers
    docker container ls --all 
    # Parar el container
    docker stop 018 
    # Borrar el container
    docker rm 018 
\end{lstlisting}

\begin{figure}[H]
    \centering
    \includegraphics[width=6in]{figures/appendix-A/list-docker-containers.png}
    \caption{La instancia de docker levantada}
\end{figure}

Ahora que se ha levantado la instancia de docker con un ID 0186392cce18 y sus respectivos puertos mapeados.
Se conecta al container mediante ssh con la contraseña \textbf{tfg-ucm}

\begin{lstlisting}[language=bash, caption=Conexión ssh]
    # El puerto 22022 es el que se habia mapeado
    ssh root@localhost -p 22022
\end{lstlisting}

Una vez conectado al container, hay que instalar las siguientes herramientas

\begin{lstlisting}[language=bash, caption=Instalaciones]
    # Actualizacion por si acaso
    apt update 
    # Para descargar el repositorio en github
    apt install git
    # El servidor es un proyecto maven
    apt install maven
    # Base de datos
    apt install postgresql
    # Para usuarios
    apt install sudo
    # El servidor es un proyecto de java
    apt install openjdk-8-jdk
\end{lstlisting}

Una vez instalado todas las herramientas, descargar el proyecto del servidor,
que está en \url{https://github.com/DanielCalle/TFG-Server.git}

\begin{lstlisting}[language=bash, caption=Descarga del proyecto]
    # Se recomienda descargar el proyecto en la carpeta home
    cd /home
    git clone https://github.com/DanielCalle/TFG-Server.git
\end{lstlisting}

Antes de levantar el servicio, hay que preparar postgresql. El archivo SQL está dentro del proyecto
en el siguiente path: \url{/src/main/webapp/WEB-INF/sql/filmar_low.sql}

\begin{lstlisting}[language=bash, caption=Configuración postgresql]
    # Levantar postgresql
    service postgresql start
    # Cambiar al usuario postgres (por defecto)
    sudo -i -u postgres
    # Iniciar psql
    psql
    # Cambiar el password a 'admin'
    \password
    # Creacion de una base de datos
    create database films;
    # Salir de psql
    \q
    # Crear la estructura de base de datos
    # Anadir datos
    psql -d films -f /home/TFG-Server/src/main/webapp/
        WEB-INF/sql/filmar_low.sql
    # Salir del usuario postgres
    exit
\end{lstlisting}

Una vez configurado postgres, se procede al despliegue del servicio.

\begin{lstlisting}[language=bash, caption=Despliegue]
    # Acceder a la raiz del repositorio
    cd /home/TFG-Server
    # Empaquetar con el perfil de docker
    mvn package -P Docker
    # Dar permiso de ejecucion al script
    chmod +x server.sh
    # Ejecutar 
    ./server.sh start
    # No hace falta ejecutar los siguientes comandos
    # Para parar el servicio
    ./server.sh start
    # Para reiniciar el servicio
    ./server.sh restart
\end{lstlisting}

Ya está levantado el servicio en el puerto 8080, se puede probar con el siguiente enlace
\url{localhost:8080/films}.

\section{Heroku}
\label{app:heroku}
Para comenzar debemos crear una cuenta en \href{https://www.heroku.com/}{Heroku}.
\begin{figure}[H]
    \centering
    \includegraphics[width=6in]{figures/chapter-4/heroku_1.png}
    \caption{Crear nueva aplicación en Heroku}
    \label{fig:heroku_1}
\end{figure}
Una vez tenemos una cuenta en la plataforma creamos una nueva aplicación a la cual daremos un nombre como vemos en la figura~\ref{fig:heroku_1} y este nombre a su vez será parte de la URL pública.
\begin{figure}[H]
    \centering
    \includegraphics[width=6in]{figures/chapter-4/heroku_2.png}
    \caption{Configuración de la aplicación}
    \label{fig:heroku_2}
\end{figure}
Cuando creemos la aplicación accederemos a la configuración en la que veremos opciones como en la figura~\ref{fig:heroku_2}.
\begin{figure}[H]
    \centering
    \includegraphics[width=6in]{figures/chapter-4/heroku_3.png}
    \caption{Métodos de despliegue}
    \label{fig:heroku_3}
\end{figure}
En la figura~\ref{fig:heroku_3} vemos que nos ofrece diversos métodos de despliegue:
\begin{itemize}
    \item Heroku Git
    \item GitHub
    \item Container Registry
\end{itemize}
Elegimos la opción de GitHub que nos proporciona un despliegue fácil y rápido. Tenemos que vincular la cuenta de nuestro GitHub (Importante ser el dueño del repositorio).
\begin{figure}[H]
    \centering
    \includegraphics[width=6in]{figures/chapter-4/heroku_4.png}
    \caption{Opciones de despliegue utilizando Github}
    \label{fig:heroku_4}
\end{figure}
Una vez vinculada la cuenta de GitHub tenemos dos métodos de despliegue como vemos en la figura~\ref{fig:heroku_4}. Un método automático para desplegar los cambios de una rama del repositorio que se accionará cada vez que subamos cambios y otro método manual que se desplegará sólo cuando presionemos el botón. 
Cuando elijamos una de las opciones, la aplicación estará lista para usarse.

\section{Guía para generar un apk}
Como hemos comentado anteriormente, nuestra aplicación consta de dos partes: La parte de Realidad Aumentada y 
la parte de Android. Por ello, a la hora de generar una apk completa de la aplicación habrá que generar estas partes 
por separado.

\subsection{Generar la parte de la aplicación en Unity.}
Para generar la aplicación en Unity:
\begin{enumerate}
    \item Abrimos en Unity el proyecto. El nombre de este es TFG-Vuforia.
    \item A continuación debemos seleccionar la escena en la que se encuentra la aplicación en Unity. Esta escena se encuentra en 
    el directorio principal Assets > Scenes y su nombre es CloudRecognition.
    \begin{figure}[H]
        \centering
        \includegraphics[width=6in]{figures/Appendix-A/CapturaCloudRecognition.JPG}
        \caption{Ubicación de la escena principal}
        \label{fig:CloudRecognitionUbication}
    \end{figure}
    \item Una vez localizada la escena principal, debemos seleccionarla. Para ello la arrastraremos al panel de trabajo (situado 
    arriba a la izquierda).
    \begin{figure}[H]
        \centering
        \includegraphics[width=4in]{figures/Appendix-A/CapturaPanel.JPG}
        \caption{Panel principal}
        \label{fig:CloudRecognitionUbication}
    \end{figure}
    \item El siguiente paso es exportar este proyecto en Unity para que pueda usarlo la parte de Android. Para ello, pulsaremos 
    File > Build Settings. Aparecerá una vista como la de la Figura \ref{fig:BuildSettings}, es muy importante antes de exportar la 
    aplicación a Android tener seleccionado Android (a la izquierda). Para seleccionarlo se marca y se presiona el botón de Switch Platform.

    Una vez que nos hemos asegurado de tener seleccionada la plataforma Android ya podemos exportar el proyecto. Para ello seleccionamos la opción 
    que dice Export proyect con el Build system de Gradle, como se indica en la Figura \ref{fig:BuildSettings}.
    \begin{figure}[H]
        \centering
        \includegraphics[width=4in]{figures/Appendix-A/CapturaBuildSettings.JPG}
        \caption{Build Settings}
        \label{fig:BuildSettings}
    \end{figure}
\end{enumerate}
\subsection{Cómo unir la parte de Unity y Android}
Como solución del la sección anterior se habrá generado en la carpeta especificada un proyecto en Android llamado FilmAR.

Si fuésemos a crear un proyecto en Android desde cero, partiríamos desde este proyecto generado por Unity. 

Si lo que queremos es actualizar nuevos cambios en Unity en la aplicación de Android seguiremos los siguientes pasos:
\begin{enumerate}
    \item Accederemos a los directorios FilmAR > src > main de las dos aplicaciones (la generada por Unity con los nuevos cambios 
    y la aplicación en Android general de la aplicación).
    \item Eliminaremos la carpeta Assets de la aplicación general y la cambiaremos por la nueva carpeta Assets generada por Unity.
\end{enumerate}
De esta manera, tendremos la aplicación en Android actualizada a los nuevos cambios de la parte de Realidad Aumentada.
\subsection{Generar un APK}
Para generar un APK, simplemente tendremos que ir en Android Studio a la pestaña Build y hacer click en generar APK. De esta forma podremos 
descargar la aplicación.
