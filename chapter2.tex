% +--------------------------------------------------------------------+
% | Sample Chapter 2
% +--------------------------------------------------------------------+

\cleardoublepage

% +--------------------------------------------------------------------+
% | Replace "This is Chapter 2" below with the title of your chapter.
% | LaTeX will automatically number the chapters.
% +--------------------------------------------------------------------+

\chapter{Estado del arte}
\label{makereference2}
\section{Realidad Aumentada}
\label{makereference2.1}
\subsection{Tecnologías actuales}
\label{makereference2.1.1}
\paragraph{ARCore:}
Plataforma creada por Google para desarrollar aplicaciones de
 realidad aumentada con soporte para Android, Android NDK, iOS,
 Unity y Unreal Engine. Aunque las funcionalidades que se ofrecen
 para iOS y Unity for iOS se limitan a Cloud Anchors, los anchors
 sirven para hacer que objetos virtuales aparezcan en un lugar
 captado por la cámara de nuestro dispositivo estos son compartidos
 en la nube para que multitud de dispositivos disfruten de la misma
 experiencia, los dispositivos con iOS podrán usarlos utilizando ARKit.

Tiene una curva de aprendizaje media y con su versión 1.5 demuestra una
 estabilidad interesante respecto a su reciente creación. Cabe destacar
 que no todos los dispositivos son compatibles, esto depende de que las
 empresas que desarrollan estos dispositivos cumplan unos requisitos
 para asegurar que la experiencia con ARCore es la adecuada y de la
 versión del sistema operativo. se puede ver con más detalle en esta
 dirección \url{https://developers.google.com/ar/discover/supported-devices}.

ARCore usa tres características a través de la cámara del dispositivo:
 \begin{itemize}  
     \item {\bf Motion tracking} permite al dispositivo entender y rastrear la posición relativa del mundo.
     \item {\bf Environmental understanding} permite al dispositivo detectar el tamaño y localización de todos los tipos de superficies.
     \item {\bf Light estimation} permite al dispositivo estimar las condiciones de luz del entorno actual.
 \end{itemize}

\paragraph{ARKit:}
Podemos utilizar experiencias de realidad aumentada persistente, compartirlas entre distintos dispositivos iOS,
 detecta imágenes 2D incluso en movimiento y objetos 3D.

\paragraph{Wikitude:}
Kit de desarrollo para realidad aumentada con soporte para Android, iOS, Unity, Cordova, Xamarin (mala
 documentación y versiones obsoletas), Titanium, React Native.
Su licencia es de pago aunque hay versiones limitadas gratuitas.
Utiliza ARCore o ARKit cuando los dispositivos lo soportan y en caso que no utiliza tecnología de Wikitude
 para que el número de dispositivos compatibles sea mayor. 

\paragraph{Vuforia:}
Kit de desarrollo para realidad aumentada con soporte para Android, iOS, UWP y Unity.
Software de pago solo permite usarse gratis para pruebas.

\paragraph{ViroReact:}
Plataforma para construir aplicaciones con realidad aumentada usando React Native.
 Utiliza ARKit y ARCore para dotar a las aplicaciones de una experiencia de RA sin
 utilizar código distinto y con una curva de aprendizaje fácil. Al basarse en React
 Native que no tiene versión estable provoca problemas con las versiones de dependencias,
 configuraciones tediosas y largas compilaciones.
Es un software privativo gratuito.

\paragraph{Expo AR:}
API que permite crear aplicaciones en React Native utilizando ARKit únicamente.
Está en una fase muy inicial.




\section{Análisis de las necesidades del usuario}
\label{makereference2.2}

\section{Entrevistas}
\label{makereference2.3}

\section{Análisis de la competencia}
\label{makereference2.4}

\section{Fuentes de información}
\label{makereference2.5}

\section{Técnicas de recomendación}
\label{makereference2.6}



