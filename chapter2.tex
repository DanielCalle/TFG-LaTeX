% +--------------------------------------------------------------------+
% | Sample Chapter 2
% +--------------------------------------------------------------------+

\cleardoublepage

% +--------------------------------------------------------------------+
% | Replace "This is Chapter 2" below with the title of your chapter.
% | LaTeX will automatically number the chapters.
% +--------------------------------------------------------------------+

\chapter{Estado del arte}
\label{makereference2}
Para que esta aplicación sea usable e interactiva, en esta sección de la memoria procederemos a 
realizar un estudio de las posibles tecnologías que nos permitan llevar a cabo nuestros objetivos, 
tales como la realidad aumentada, las distintas fuentes de información y algunas
técnicas de recomendación. 
\section{Realidad Aumentada}
\label{makereference2.1}
El gran peso de nuestra aplicación está basado en la tecnología de la realidad aumentada que, tras investigar su significado en varias fuentes de información,
decidimos definirla de la siguiente forma:
\begin{quote}
``
La realidad aumentada nos permite añadir capas de información visual sobre el 
mundo real que nos rodea, utilizando la tecnología, dispositivos como pueden ser 
nuestros propios teléfonos móviles. Esto nos ayuda a generar experiencias que aportan
un conocimiento relevante sobre nuestro entorno, y además recibimos esa información en 
tiempo real. Mediante la realidad aumentada el mundo virtual se entremezcla con el mundo 
real, de manera contextualizada, y siempre con el objetivo de comprender mejor todo lo que 
nos rodea.
''
\end{quote}


\begin{figure}[htb]
    \centering
    \makebox[0pt][c]{%
    \begin{minipage}[b]{0.5\linewidth}
    \centering
      \includegraphics[scale=0.2]{figures/chapter-2/televisionAR-1.jpg}
      \caption{Telemadrid anuncia el uso de RA en las elecciones de 2019}
      \label{fig:sva}
    \end{minipage}%
    \hspace{0.3cm}
    \begin{minipage}[b]{0.5\linewidth}
    \centering
     \includegraphics[scale=0.3]{figures/chapter-2/televisionAR-3.jpg}
      \caption{Antena 3 usa RA para las elecciones del ayuntamiento de Madrid}
    \label{fig:svb}
    \end{minipage}%
    }%
\end{figure}


La realidad aumentada se aplica a diversas áreas como seguridad, logística, sanidad, ocio, etc...
 Se puede aplicar a tantas disciplinas como podamos imaginar y no sólo se ha convertido en una herramienta útil, sino que es una tecnología atractiva
 y además signo de modernidad. Por estas razones actualmente vemos ejemplos como el de las televisiones, que compiten por aplicarla a sus programas
 y presumir de dar el mejor servicio a sus espectadores como vemos en las Figuras~\ref{fig:sva} y \ref{fig:svb}. Con esta situación podemos por tanto estar de acuerdo
 en que es una tecnología en auge actualmente. 


Los smartphones son los dispositivos que más interaccionan con esta tecnología, ya que prácticamente todas las personas poseen uno
 y están dotados de un amplio abanico de sensores necesarios para su correcto funcionamiento. Por esto elegimos este tipo de dispositivos para nuestro
 proyecto ya que, esta tecnología, aporta un gran valor añadido a nuestra aplicación, incluso siendo el grueso de la misma. 

 La realidad aumentada en cualquier dispositivo se basa en dónde nos encontramos y hacia donde estamos mirando, es decir, si usamos la cámara del teléfono,
 según a donde estemos mirando a través de ésta, la realidad aumentada superpondrá objetos virtuales o información a nuestra realidad, con un cierto tamaño y una cierta posición
 basándose en las medidas de lo que vemos a través de nuestra cámara. Por ejemplo, la famosa aplicación de Pokemon Go, que superpone la imagen de un pokemon a la realidad
 que vemos con la cámara para que podamos atraparlo, haciendo así que su uso sea más interactivo y visual para el usuario, como podemos observar en la Figura \ref{fig:pokemon}. En las Figuras \ref{fig:libertystatue} y \ref{fig:measure}
 podemos ver otros ejemplos del uso de la realidad aumentada en dispositivos móviles, como el reconocimiento de monumentos turísticos o para medir ciertas distancias.

 \begin{figure}[H]
    \centering
    \includegraphics[width=4in]{figures/chapter-2/pokemongo.png}
    \caption{RA en Pokemon Go\cite{pokemongo}}
    \label{fig:pokemon}
\end{figure}

\begin{figure}[htb]
    \centering
    \makebox[0pt][c]{%
    \begin{minipage}[b]{0.5\linewidth}
    \centering
      \includegraphics[scale=0.3]{figures/chapter-2/ralibertystatue.png}
      \caption{RA para zonas turísticas\cite{ralibertystatue}}
      \label{fig:libertystatue}
    \end{minipage}%
    \hspace{0.3cm}
    \begin{minipage}[b]{0.5\linewidth}
    \centering
     \includegraphics[scale=0.3]{figures/chapter-2/rameasure.jpg}
      \caption{RA para mediciones\cite{rameasure}}
    \label{fig:measure}
    \end{minipage}%
    }%
\end{figure}

\subsection{Tecnologías actuales}
\label{makereference2.1.1}
A continuación, describiremos las distintas tecnologías que hemos investigado
 para la implementación de la Realidad Aumentada.
\begin{itemize}
    \item ARCore\cite{arcore}: esta plataforma creada por Google para
    desarrollar aplicaciones de realidad aumentada con soporte para Android,
    Android NDK, iOS, Unity y Unreal Engine, aunque las funcionalidades que
    se ofrecen para iOS y Unity for iOS se limitan a Cloud Anchors.
    Los anchors son herramientas que permiten que objetos virtuales
    aparezcan en un lugar captado por la cámara de nuestro dispositivo,
    estos son compartidos en la nube para que multitud de dispositivos
    disfruten de la misma experiencia, aquellos que posean iOS podrán usarlos
    utilizando ARKit.

    ARCore usa tres características a través de la cámara del dispositivo:
    \begin{itemize}  
        \item Motion tracking: permite al dispositivo conocer la posición
        relativa del mundo.
        \item Environmental understanding: permite al dispositivo detectar el
        tamaño y localización de todos los tipos de superficies.
        \item Light estimation: permite al dispositivo estimar las condiciones
        de luz del entorno actual.
    \end{itemize}
    Puntos positivos:
    \begin{itemize}
        \item Buena integración con Android.
        \item Tiene una curva de aprendizaje media y con su versión 1.5
        demuestra una estabilidad interesante respecto a su reciente creación.
    \end{itemize}
    Puntos negativos:
    \begin{itemize}
        \item Poca documentación para soluciones más complejas.
        \item No todos los dispositivos son compatibles.
        Esto depende de que las empresas que desarrollan estos dispositivos
        cumplan unos requisitos para asegurar que la experiencia con ARCore
        es la adecuada y de la versión del sistema operativo.
    \end{itemize}

    \item ARKit\cite{arkit}: es una librería desarrollada por Apple que
    podemos utilizar para crear experiencias de realidad aumentada persistente y
    compartirlas entre distintos dispositivos iOS. 
    Esta tecnología se basa en la odometría visual inercial,
    esta es capaz de reconocer las imágenes que capta la cámara y la forma en
    la que la luz se refleja en los diferentes elementos que aparecen para
    obtener un mapa 3D del entorno y, calcular las distancias que hay entre los
    diferentes objetos desde la posición del dispositivo.
    Tras iniciar la realidad aumentada con ARKit,
    nuestro dispositivo crea un entorno virtual donde nuestro
    dispositivo representa la coordenada (0,0,0), con los tres ejes,
    el horizontal, vertical y el de profundidad, pero también es necesario
    otro eje, el eje W que representa la rotación del dispositivo. En la Figura
    \ref{fig:arkit} se puede observar cómo cambia el elemento virtual según como
    situemos nuestro smartphone.

    \begin{figure}[H]
        \centering
        \includegraphics[width=4in]{figures/chapter-2/arkit.png}
        \caption{RA en ARKit\cite{arkitimage}}
        \label{fig:arkit}
    \end{figure}

    Puntos positivos:
    \begin{itemize}
        \item Detecta imágenes 2D incluso en movimiento y objetos 3D.
        \item Buena integración con iOS.
    \end{itemize}
    Puntos negativos:
    \begin{itemize}
        \item Es necesario ordenadores con macOS para el desarrollo.
    \end{itemize}


    \item Wikitude\cite{wikitude}:
    es un kit de desarrollo para realidad aumentada con soporte para Android,
    iOS, Unity, Cordova, Xamarin, Titanium y React Native.
    Utiliza ARCore o ARKit cuando los dispositivos lo soportan y en caso
    contrario utiliza tecnología de Wikitude para que el número de dispositivos
    compatibles sea mayor. Algunas de las tecnologías que soporta Wikitude son
    la geolocalización, el reconocimiento de imágenes y el reconocimiento en la
    nube. Para entender cómo funciona este SDK tenemos que describir tres
    términos básicos fundamentales:
    \begin{itemize}
        \item Target: conjunto de datos extraídos de una imagen, empleados por
        un rastreador al reconocerla para realizar alguna acción.
        \item Target Collection: es un conjunto de targets empleado por el
        rastreador para reconocer imágenes en el mundo real.
        \item Client Tracker: rastreador que usa la cámara para analizar objetos
        2D, analiza el Target Collection y busca similitudes con la imagen
        reconocida.
    \end{itemize}

    Puntos positivos:
    \begin{itemize}
        \item Compatibilidad con muchos dispositivos.
        \item Dispone de reconocimiento de imágenes en la nube.
    \end{itemize}
    Puntos negativos:
    \begin{itemize}
        \item Su licencia es de pago, aunque hay versiones limitadas gratuitas,
        pero limitan su funcionalidad.
        \item Documentación obsoleta para algunas plataformas de desarrollo.
    \end{itemize}


    \item Vuforia\cite{vuforia}:
    Kit de desarrollo para realidad aumentada con soporte para Android, iOS,
    UWP y Unity.
    Su integración con Unity permite crear aplicaciones y juegos 
    para que sean usados en Android y iOS simplemente arrastrando y soltando
    elementos. Usa un seguimiento basado en marcadores, los marcadores son las
    imágenes u objetos reconocidos por la aplicación para desencadenar la
    visualización del contenido virtual sobre su posición en la vista de la
    cámara.
    Los Image Targets u objetivos de imagen, son un tipo específico de marcador,
    son imágenes que se registran manualmente en la aplicación y que pueden ser
    guardadas en un Cloud, reduciendo así el peso de tener las imágenes a
    reconocer en el dispositivo.
    También se puede realizar un seguimiento sin marcado, basado en el GPS o en
    un giroscopio para insertar modelos de objetos virtuales en la realidad que
    capta la cámara.

    Puntos positivos:
    \begin{itemize}
        \item Se encuentran fácilmente ejemplos y documentación sobre su uso.
        \item Compatibilidad con muchos dispositivos.
        \item Dispone de reconocimiento de imágenes en la nube.
    \end{itemize}
    Puntos negativos:
    \begin{itemize}
        \item Su licencia es de pago, aunque puede usarse gratis para desarrollar y
        probar. Con límites en su funcionalidad.
    \end{itemize}


    \item ViroReact\cite{viroreact}:
    Plataforma para construir aplicaciones con realidad aumentada usando
    React Native. Es capaz de combinar tecnologías web que se pueden utilizar
    en distintas plataformas con llamadas a APIs internas de las distintas
    plataformas móviles. De este modo, podemos generar aplicaciones fácilmente
    que funcionen en Android utilizando el potencial de ARCore y en iOS
    utilizando el de ARKit.
    A pesar de ser software privativo, su licencia no tiene limitaciones en
    las funcionalidades.

    Puntos positivos:
    \begin{itemize}
        \item Curva de aprendizaje fácil.
        \item Se utiliza el mismo código para generar aplicaciones en iOS y
        Android.
    \end{itemize}
    Puntos negativos:
    \begin{itemize}
        \item Al basarse en React Native, que no tiene versión estable,
        provoca problemas con las versiones de dependencias.
        \item Configuraciones tediosas y largas compilaciones.
    \end{itemize}


    \item Expo AR\cite{expoar}:
    Es un conjunto de librerías escritas de forma nativa para cada plataforma,
    proporcionando acceso a la funcionalidad del sistema del
    dispositivo (como pueden ser la cámara, notificaciones emergentes,
    contactos, almacenamiento local y otro tipo de hardware) desde JavaScript.
    Está desarrollado para suavizar las diferencias entre plataformas lo máximo
    posible, consiguiendo que los proyectos que se realicen sean muy portables
    ya que pueden ejecutarse en cualquier entorno nativo que contengo el SDK de
    Expo.
    También proporciona componentes de interfaz de usuario para manejar una
    variedad de casos de uso que casi todas las aplicaciones móviles tienen que
    cubrir pero no están construidas en el núcleo de React Native, por ejemplo,
    iconos, vistas borrosas (aquellas que aparecen de fondo para que el usuario
    se centre más en los elementos que están superpuesto a la imagen), etc.

    Puntos positivos:
    \begin{itemize}
        \item Curva de aprendizaje fácil.
    \end{itemize}
    Puntos negativos:
    \begin{itemize}
        \item En el momento del desarrollo de este proyecto solo tiene soporte
        para iOS con ARKit.
        \item Al basarse en React Native que no tiene versión estable
        provoca problemas con las versiones de dependencias.
        \item Configuraciones tediosas y largas compilaciones.
    \end{itemize}
\end{itemize}


\section{Fuentes de información}
\label{makereference2.2}
Como nuestra aplicación muestra información sobre las películas que se recomiendan y que los usuarios han guardado, hemos
tenido que investigar de qué fuentes de información podríamos obtener estos datos tan relevantes. Con la finalidad de extraer
estos datos, guardarlos en contenedores de datos y por último, representarlos ante los usuarios.
\begin{itemize}
    \item FilmAffinity\cite{filmaffinity}: es una de las páginas webs más relevantes para obtener información sobre películas y series, además de críticas y valoraciones de profesionales y sugerencias personalizadas.
    Esta página fue creada con el objetivo de ser un sistema de recomendación, siendo un factor que nos beneficia ya que podemos estar seguros de que sus datos nos servirán para realizar recomendaciones a los usuarios.
    Por otro lado, la página está en dos idiomas, español e inglés, otro factor a favor ya que nuestra aplicación muestra los datos en inglés. Sin embargo, al ser una página en la 
    que se realizan críticas serias y profesionales, las valoraciones suelen ser bajas.
    \item MovieLens\cite{movielens}: es una fuente de información, quizás menos conocida que FilmAffinity y que IMDB pero que ofrece sugerencias personalizadas y la posibilidad de valorar. Se trata de un sistema de recomendación basado en una técnica de recomendación 
    llamada filtrado colaborativo\cite{filtradocolaborativo}, técnica candidata a usar en nuestra aplicación y de la que hablaremos posteriormente. Los datos que nos proporciona están en inglés, valiéndonos perfectamente para nuestra aplicación.
    \item IMDB\cite{imdb}: se trata de una base de datos online donde se guarda información sobre películas y series. Es una de las páginas más usadas a nivel mundial en estos aspectos.
    Actualmente, le pertenece a una de las empresas que más clouds de almacenamiento tiene en el mundo, Amazon, por lo que podemos asegurar su fiabilidad. El idioma de la página también es el inglés.
    \item Rotten Tomatoes\cite{rottentomatoes}: es una web donde usuarios y
     críticos profesionales puntúan películas y espectáculos televisivos.
    La puntuación es representada con tomates frescos si el 60\% de las
     críticas son positivas y si es menor de ese porcentaje es representada con
     un tomate podrido explotando.
    Existen certificados de una distinción especial si además cumplen más
     requisitos:
     \begin{itemize}
         \item Una puntuación constante en el Tomatómetro de 75\% o más.
         \item Al menos cinco reseñas de los mejores críticos.
         \item Las películas en ``wide release'' deben tener un mínimo de 80 revisiones.
         \item Las películas en ``limited release'' deben tener un mínimo de 40 revisiones.
         \item Solo las temporadas individuales de un programa de televisión son certificables, y cada uno ha de tener un mínimo de 20 revisiones.
     \end{itemize}
    Estos certificados además de que tienen que ser valorados por un grupo de jurados, pueden además ser retirados una vez que esta película o serie
    no sea capaz de mantener los requisitos.
\end{itemize}

Finalmente, hemos elegido IMDB como fuente de información para nuestra aplicación, ya que es una de las páginas webs más conocidas mundialmente. Su fiabilidad está asegurada y
además estamos más familiarizados con ella ya que es la que usamos normalmente en nuestro día a día.

\section{Técnicas de recomendación}
\label{makereference2.3}
Comenzaremos esta sección definiendo qué es un sistema de recomendación y para qué se usan.
Los sistemas de recomendación (SRs)\cite{HandbookRS} son herramientas software y técnicas que proporcionan sugerencias de productos útiles para un usuario. 
Las sugerencias están relacionadas a varios procesos de toma de decisiones
como pueden ser productos que comprar, qué música escuchar, o que noticias online podrías leer.
Un SR normalmente se centra en un tipo específico de producto (por ejemplo, CDs o noticias) 
y de acuerdo a su diseño, su interfaz gráfica de usuario, y el núcleo de la 
técnica de recomendación usada para generar recomendaciones están todas personalizadas para proveer de sugerencias útiles y efectivas para ese tipo específico de objeto.


Hoy en día, con toda la información que dejamos en internet, los sistemas de recomendación comienzan a ganar popularidad.
Muchas empresas mundialmente conocidas nos ofrecen recomendaciones de sus productos en base a un análisis de nuestros datos y comportamientos.
Por ejemplo Youtube, que registra los últimos videos que hemos visto y nos recomienda vídeos similares, basándose en el contenido de dichos vídeos. Otro caso es Netflix, que recomienda películas 
en base a factores como las interacciones de los usuarios en la página web o de usuarios con gustos similares\cite{netflixrecommendation}. Es una de las empresas pioneras en el sector 
de las recomendaciones junto con Amazon. Netflix además tiene numerosas técnicas de recomendación debido a un concurso 
que hizo en 2009, ofreciendo un millón de dólares a quienes mejor optimizaran sus sistemas de recomendación\cite{netflixprize}.

Los sistemas de recomendación se pueden dividir según su forma de realizar las recomendaciones:

\begin{itemize}
    \item Filtrado basado en el contenido\cite{filtradocontenido}: al usuario se le recomendará aquella información que le ha interesado en el pasado, es decir, se le mostrarán elementos similares, independientemente de lo que opinen otros usuarios. 
    Se basa en contenido de los objetos, se analiza el objeto para sacar características relevantes de ello, y a partir de las características 
    extraídas, recomendar otros objetos con características similares.   
    \item Filtrado Demográfico\cite{filtrademografico}: al contrario que el filtrado basado en el contenido que se basa en los objetos, este se basa en el usuario,
    se realiza en función de las características que tienen los usuarios, como edad, sexo, profesión, etc. Ya que se asume que usuarios que, por ejemplo,
    han nacido en la misma década van a tener preferencias similares frente a los de otras décadas. O que a los profesores de matemáticas les gustan más los números 
    que a los profesores de filología. 
    \item Filtrado Colaborativo\cite{filtradocolaborativo}: esta técnica busca similitudes de preferencias entre usuarios, y recomienda objetos 
    que le hayan gustado a otros usuarios similares. La determinación de usuarios con preferencias similares se basa en 
    las valoraciones que hayan dado los usuarios a los objetos.
    \item Filtrado Híbrido\cite{filtrahibrido}: Mezcla las tres técnicas descritas anteriormente, aprovechando los aspectos positivos
    que ofrece cada una. Incluso se combina con técnicas de inteligencia artificial como la lógica borrosa o la computación evolutiva. 
\end{itemize} 

\begin{figure}[H]
    \centering
    \includegraphics[width=6in, angle=0]{figures/chapter-2/recommendation_systems.png}
    \caption{Tipos de sistemas de recomendación \cite{imagefilters}}
\end{figure}

En nuestra aplicación, los objetos serán las películas que almacenaremos en nuestro contenedor de datos. Nos basaremos en 
la técnica de filtrado colaborativo, ya que permitimos a los usuarios la posibilidad de 
interactuar en el sistema de recomendación valorando las películas que les gusten. Otro motivo es que, tanto el filtrado basado
en el contenido como el filtrado demográfico, necesitan retener información extra sobre los usuarios, como los objetos, representados 
en forma de etiquetas, para finalmente llevar a cabo una recomendación.

Para la implementación de los sistemas de recomendación hemos encontrados tres librerías:
\begin{enumerate}
    \item Mahout: Se trata de una librería que abarca la mayoría de necesidades para la implementación del sistema de recomendación.
    \item Lenskit: Es otra librería de Java para la implementación de sistemas de recomendación.
    \item LibRec: Se trata de una librería que está subida en un repositorio de GitHub.
\end{enumerate}

Al final nos hemos decantado por Mahout, ya que aporta una documentación más completa. Otro de los motivos es que ofrece clases para realizar la conexión directamente con la base de datos 
sin tener que leer desde ficheros csv. Por otro lado, en el testeo de 
Lenskit, cuando intentábamos reproducir el ejemplo que ofrecían, nos encontramos que algunas funciones estaban 
obsoletas y, por tanto, no nos transmitía mucha confianza. Por último, Mahout destaca por ser más fácil de usar y minimizar nuestro esfuerzo en aprender esta nueva tecnología.

\section{Conclusiones}
\label{makereference2.4}
En este capítulo hemos podido conocer de primera mano la tecnología que representa el grueso de nuestra aplicación, la realidad aumentada. Se han descrito 
los distintos tipos de librerías que hemos investigado para su implementación, así como las diferentes páginas web de las que podríamos obtener la información necesaria
para nutrir a nuestra aplicación y a nuestros usuarios. Finalmente hemos incluido un breve resumen de las técnicas de recomendación que más se ajustan a nuestra aplicación. En los próximos capítulos
se tratará más a fondo este tema, cuando se haya descrito con más detalle nuestra aplicación.

