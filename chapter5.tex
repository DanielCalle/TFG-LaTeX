% +--------------------------------------------------------------------+
% | Sample Chapter 5
% +--------------------------------------------------------------------+

\cleardoublepage

% +--------------------------------------------------------------------+
% | Replace "This is Chapter 5" below with the title of your chapter.
% | LaTeX will automatically number the chapters.
% +--------------------------------------------------------------------+

\chapter{Conclusiones/Conclusions}
\label{makereference5}

\section{Conclusiones}
\label{makereference5.1}
Cuando comenzamos a hablar sobre este proyecto, lo que nos motivó fue el
 reto de aprender nuevas tecnologías desconocidas para nosotros y afrontarlas
 en un proyecto que soluciona un problema concreto. La solución a este problema
 es una aplicación móvil que nos ofrece la facilidad de crear y gestionar
 planes con amigos para ir al cine. Usando un sistema de recomendación somos
 capaces de tener una previsión sobre lo que nos podría gustar la película en
 base a otras que hayamos valorado anteriormente.
 Con el uso de tecnología de RA buscamos que el usuario se sienta más cómodo
 utilizando la aplicación sin tener que introducir muchos datos, si no que el
 propio dispositivo móvil sea capaz de reconocer los carteles de películas o
 imágenes de usuarios y nos ofrezca acciones sobre ello.

Al principio buscamos aprender los aspectos más básicos que nos permitieran
 diseñar e implementar de la aplicación. Para ello realizamos un estudio
 del estado del arte en el que desempeñamos las siguientes tareas:
\begin{itemize}  
    \item Buscamos información y explicamos que es la Realidad Aumentada y cuál
     es su estado actual. Describimos diferentes librerías de Realidad Aumentada
     con sus aspectos más destacables.
    \item Descubrimos y especificamos distintas fuentes de información de las
     cuales extraer los datos necesarios sobre las películas de la aplicación
     que se mostrara o se utilizaran para el sistema de recomendación.
    \item Buscamos información sobre los distintos tipos de técnicas de
     recomendación y definimos las más interesantes.
\end{itemize}

En el siguiente paso, diseñamos la aplicación y definimos su funcionalidad,
 realizando las siguientes tareas:
\begin{itemize}
    \item Realizamos un análisis de competencia en el cual descubrimos
     aplicaciones similares que nos ayudaron a visualizar el diseño de nuestra
     aplicación.
    \item Definimos escenarios donde se podría usar nuestra aplicación.
    \item En base a los escenarios anteriormente definidos, establecimos los
     requisitos funcionales que describen la funcionalidad de la aplicación.
    \item Diseñamos las interfaces de la aplicación para las funcionalidades
     anteriormente comentadas.
\end{itemize}

Posteriormente, pasamos a la parte de implementación donde realizamos las
 siguientes tareas:
\begin{itemize}
    \item Desarrollamos prototipos con distintas librerías de realidad
     aumentada en aplicaciones móviles y valoramos la experiencia con ellas.
    \item Definimos una arquitectura REST para nuestra aplicación y un
     reconocimiento de imágenes en la nube.
    \item Implementamos un servidor en Spring desplegado en Heroku que usa como
     base de datos PostgreSQL.
    \item Implementamos la aplicación móvil en Android, la parte de RA en Unity
     con la librería Vuforia.
    \item Valoramos distintos sistemas de recomendación y finalmente lo
     implementamos con Mahout.
\end{itemize}

Finalmente, fuimos capaces de aplicar los conceptos generales que aprendimos
 en el grado en un entorno nuevo y desafiante, gracias al valor humano de
 un equipo multidisciplinar en el cual los integrantes complementan sus
 puntos débiles individuales. 

\section{Conclusions}
\label{makereference5.2}
At the beginning, the fact of learning new technologies and facing them
to solve a specific problem was the main issue that encouraged us for this project. 
Our solution to this problem is a mobile application that offers us the ease of 
creating and managing movie-based plans with friends. We are able to get predictions 
about the likeliness of enjoyment for a movie by using a recommendation system, based on 
a historical film valorations list. We are looking for the ease of using this application without 
having to introduce many data, the application is capable of recognizing movie posters or
user images by itself and apply actions.

Al principio buscamos aprender los aspectos más básicos que nos permitieran
 diseñar e implementar de la aplicación. Para ello realizamos un estudio
 del estado del arte en el que desempeñamos las siguientes tareas:
\begin{itemize}  
    \item Buscamos información y explicamos que es la Realidad Aumentada y cuál
     es su estado actual. Describimos diferentes librerías de Realidad Aumentada
     con sus aspectos más destacables.
    \item Descubrimos y especificamos distintas fuentes de información de las
     cuales extraer los datos necesarios sobre las películas de la aplicación
     que se mostrara o se utilizaran para el sistema de recomendación.
    \item Buscamos información sobre los distintos tipos de técnicas de
     recomendación y definimos las más interesantes.
\end{itemize}

En el siguiente paso, diseñamos la aplicación y definimos su funcionalidad,
 realizando las siguientes tareas:
\begin{itemize}
    \item Realizamos un análisis de competencia en el cual descubrimos
     aplicaciones similares que nos ayudaron a visualizar el diseño de nuestra
     aplicación.
    \item Definimos escenarios donde se podría usar nuestra aplicación.
    \item En base a los escenarios anteriormente definidos, establecimos los
     requisitos funcionales que describen la funcionalidad de la aplicación.
    \item Diseñamos las interfaces de la aplicación para las funcionalidades
     anteriormente comentadas.
\end{itemize}

Posteriormente, pasamos a la parte de implementación donde realizamos las
 siguientes tareas:
\begin{itemize}
    \item Desarrollamos prototipos con distintas librerías de realidad
     aumentada en aplicaciones móviles y valoramos la experiencia con ellas.
    \item Definimos una arquitectura REST para nuestra aplicación y un
     reconocimiento de imágenes en la nube.
    \item Implementamos un servidor en Spring desplegado en Heroku que usa como
     base de datos PostgreSQL.
    \item Implementamos la aplicación móvil en Android, la parte de RA en Unity
     con la librería Vuforia.
    \item Valoramos distintos sistemas de recomendación y finalmente lo
     implementamos con Mahout.
\end{itemize}

Finalmente, fuimos capaces de aplicar los conceptos generales que aprendimos
 en el grado en un entorno nuevo y desafiante, gracias al valor humano de
 un equipo multidisciplinar en el cual los integrantes complementan sus
 puntos débiles individuales. 