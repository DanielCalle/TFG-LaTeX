% +--------------------------------------------------------------------+
% | Sample Chapter 5
% +--------------------------------------------------------------------+

\cleardoublepage

% +--------------------------------------------------------------------+
% | Replace "This is Chapter 5" below with the title of your chapter.
% | LaTeX will automatically number the chapters.
% +--------------------------------------------------------------------+

\chapter{Conclusiones}
\label{makereference5}
Cuando comenzamos a hablar sobre este proyecto, lo que nos motivo fue el
 reto de aprender nuevas tecnologías desconocidas para nosotros y afrontarlas
 en un proyecto real.  

Al principio aprendimos los aspectos más básicos dentro del estudio del
 estado del arte. Conocimos hasta donde abarca la \textbf{realidad aumentada}
  y que tecnologías las aplicaban cada una con sus ventajas y sus
 limitaciones. Esto nos permitió hacer una lista de las tecnologías
 que mejor se adaptaban a nuestro proyecto. Realizamos un análisis de
 competencia donde observamos algunas aplicaciones que se aproximaban la
 idea de la aplicación que queríamos construir. Buscamos las fuentes de
 información que necesitábamos y entendimos mejor que datos eran valiosos.
 Otro punto fue conocer las distintas técnicas de recomendación para entender como incluirlas en nuestra aplicación. 

En el siguiente paso, dentro del diseño de la aplicación nos centramos en
 cómo sería nuestra aplicación. Exploramos que grupos de interés habría y
 los escenarios donde se usaría. A partir de ese punto fuimos capaces de
 definir los requisitos funcionales que tendría. Investigamos los sistemas
 de recomendación que se ajustaban a las características de la aplicación,
 aplicamos las tecnologías de \textbf{realidad aumentada} anteriormente
 seleccionadas en prototipos con las características básicas que
 necesitábamos.Como ultimo diseñamos y comprendimos las interfaces que
 tendría nuestra aplicación. 

Posteriormente, pasamos a la parte de implementación donde definimos
 inicialmente la arquitectura. Con esto fuimos capaz de implementar
 la parte del servidor en \textbf{Spring} desplegándola en \textbf{Heroku}
 y la aplicación de \textbf{Android} con escenas integradas de \textbf{Vuforia} y
 \textbf{Unity}. Todo esto fue un gran reto donde encontramos dificultades,
 pero también herramientas que nos ayudaron a avanzar en nuestra meta. 

Finalmente, fuimos capaces de aplicar los conceptos generales que aprendimos
 en el grado en un entorno nuevo y desafiante, gracias al valor humano de
 un equipo multidisciplinar en el cual los integrantes complementan sus
 puntos débiles individuales. 

\section{Conclusiones}
\label{makereference5.1}

\section{Conclusions}
\label{makereference5.2}