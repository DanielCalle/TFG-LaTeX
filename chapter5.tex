% +--------------------------------------------------------------------+
% | Sample Chapter 5
% +--------------------------------------------------------------------+

\cleardoublepage

% +--------------------------------------------------------------------+
% | Replace "This is Chapter 5" below with the title of your chapter.
% | LaTeX will automatically number the chapters.
% +--------------------------------------------------------------------+

\chapter{Conclusiones/Conclusions}
\label{makereference5}

El objetivo de nuestro proyecto se basa en realizar explicaciones mediante realidad aumentada sobre 
carteles de películas y caras de usuarios. La información que mostramos debe ser relevante y entendible para el 
usuario, tanto para las películas como para los usuarios de la aplicación que se reconozcan con la cámara del dispositivo. 
Para lograr esto hemos implementado una aplicación móvil completa que nos permitiese realizar dicha función.

Cuando comenzamos a hablar sobre este proyecto, lo que nos motivó fue el
 reto de aprender nuevas tecnologías desconocidas para nosotros y afrontarlas
 en un proyecto que soluciona un problema concreto. La solución a este problema
 es una aplicación móvil que nos ofrece la facilidad de crear y gestionar
 planes con amigos para ir al cine. Usando un sistema de recomendación somos
 capaces de tener una previsión sobre lo que nos podría gustar la película en
 base a otras que hayamos valorado anteriormente.
 Con el uso de tecnología de RA buscamos que el usuario se sienta más cómodo
 utilizando la aplicación, siendo 
 propio dispositivo móvil capaz de reconocer los carteles de películas, o las
 imágenes de usuarios, ofreciéndonos la posibilidad de realizar acciones sobre éstas.

Al principio buscamos aprender los aspectos más básicos que nos permitieran
 diseñar e implementar la aplicación. Para ello realizamos un estudio
 del estado del arte en el que desempeñamos las siguientes tareas:
\begin{itemize}  
    \item Buscamos información y explicamos que es la Realidad Aumentada y cuál
     es su estado actual. Describimos diferentes librerías de Realidad Aumentada
     con sus aspectos más destacables.
    \item Descubrimos y especificamos distintas fuentes de información de las
     cuales extraer los datos necesarios sobre las películas de la aplicación,
     que se mostrarán o se utilizarán para el sistema de recomendación.
    \item Buscamos información sobre los distintos tipos de técnicas de
     recomendación y definimos las más interesantes.
\end{itemize}

En el siguiente paso, diseñamos la aplicación y definimos su funcionalidad,
 realizando las siguientes tareas:
\begin{itemize}
    \item Realizamos un análisis de competencia en el cual descubrimos
     aplicaciones similares que nos ayudaron a visualizar el diseño de nuestra
     aplicación.
    \item Definimos escenarios donde se podría usar nuestra aplicación.
    \item En base a los escenarios anteriormente definidos, establecimos los
     requisitos funcionales que describen la funcionalidad de la aplicación.
    \item Diseñamos las interfaces de la aplicación para las funcionalidades
     anteriormente comentadas.
\end{itemize}

Posteriormente, pasamos a la parte de implementación donde realizamos las
 siguientes tareas:
\begin{itemize}
    \item Desarrollamos prototipos con distintas librerías de realidad
     aumentada en aplicaciones móviles y valoramos la experiencia con ellas.
    \item Definimos una arquitectura REST para nuestra aplicación y un
     reconocimiento de imágenes en la nube.
    \item Implementamos un servidor en Spring desplegado en Heroku que usa como
     base de datos PostgreSQL.
    \item Implementamos la aplicación móvil en Android y la parte de RA en Unity
     con la librería Vuforia.
    \item Valoramos distintos sistemas de recomendación y finalmente lo
     implementamos con Mahout.
\end{itemize}

Finalmente, fuimos capaces de aplicar los conceptos generales que aprendimos
 en el grado en un entorno nuevo y desafiante, gracias al valor humano de
 un equipo multidisciplinar en el cual los integrantes complementan sus
 puntos débiles individuales. 

Our project$'$s objective is based on carrying out explanations through augmented reality about movie posters and user$'$s faces. 
The information that we show must be relevant and understandable for the user, both for the movies and for the users of the application which are
recognized by the device$'$s camera. To achieve this we implemented a complete mobile app that allowed us to perform this functionality.
When we started talking about this project, the fact of learning new technologies and facing them
to solve a specific problem was the main issue that encouraged us. 
Our solution to this problem is a mobile application that offers us the ease of 
creating and managing movie-based plans with friends. We are able to get predictions 
about the likeliness of enjoyment for a movie by using a recommendation system, based on 
a historical film valorations list. Using Augmented Reality, our objective is that a 
user feels more comfortable using this application, being the mobile device capable of recognizing 
movie posters, or user images, offering us the possibility of performing actions on these.

At the beginning we seek to learn the most basic aspects that would allow us to design
and implement the application. In order to do so we carried out a study of the state of art where we did the following tasks:
\begin{itemize}  
    \item We searched for information and explained what Augmented Reality is and what is it$'$s current state. We described different
     Augmented Reality libraries with their most remarkable aspects.
    \item We discovered and specificated different information sources from which we could extract the necessary data about the films in our application, which will be showed or used for the recommendation system.
    \item We searched for information about the different types of recommendation techniques and we defined the most interesting ones.
\end{itemize}

In the next step, we designed the application and defined it$'$s functionality, doing the following tasks:
\begin{itemize}
    \item We made a competition analysis in which we discovered similar applications to ours which helped us visualize our application$'$s design.
    \item We defined scenarios where our application could be used.
    \item Based on those scenarios previously defined, we established the functional requirements which describe the application$'$s functionality.
    \item We designed the interfaces of the application for the functionalities previously commented.
\end{itemize}

Later, we started with the implementation part, where we did the following tasks:
\begin{itemize}
    \item We developed prototypes with different augmented reality libraries for mobile devices and we valued our experience with them.
    \item We defined a REST arquitecture for our application and image recognition on the cloud.
    \item We implemented a server on Spring deployed on Heroku which uses PostgreSQL as a database.
    \item We implemented the mobile application on Android and the AR part on Unity using Vuforia$'$s library.
    \item We valorated different recommendation systems and finally implemented one using Mahout.
\end{itemize}

Finally, we were capable of applying the general concepts that we learned in our degree in a new and challenging environment, thanks to the human value of a multidisciplinary team in which the members complement each others individual weak points.

\section{Trabajo futuro}
\label{makereference5.2}
Uno de los mayores problemas del proyecto fue la dificultad de integrar
 distintas tecnologías entre sí y que estas cumplieran con todos los requisitos
 que buscábamos. Esto ocasiono que el tiempo de aprendizaje y adaptación del uso
 de estas tecnologías se disparará, por tanto, a las primeras funcionalidades que
 teníamos en mente tuvimos que priorizar las que daban el uso básico al problema
 que planteábamos.
A continuación, destacaremos todas esas funcionalidades o mejoras que nos
 gustaría que tuviera en un futuro la aplicación para que la usabilidad y
 posibilidades sean más ricas para el usuario:
\begin{enumerate}
    \item Una de las situaciones que se pueden encontrar nuestros usuarios es
     que quieren ir al cine a ver una película con amigos, pero no les importe
     que película si no pasar un buen rato con buena compañía. En este caso la
     aplicación podría crear planes y una vez se unan todos los amigos sea capaz
     de recomendar la película que mejor se ajusta a sus preferencias.
    \item Otra situación similar a la anterior es que queramos que nos
     recomiende la película que mejor se adapta a los gustos del grupo, pero
     acotando a las películas que elijamos previamente.
    \item En el momento del desarrollo de este proyecto el reconocimiento facial
     no está integrado en las librerías de RA que hemos investigado y la
     complejidad de integrar tecnologías de reconocimiento facial, así como su
     aprendizaje no era viable para el tiempo de este proyecto.
    Con lo que se decidió realizar el reconocimiento del usuario con imágenes
     asemejando esta funcionalidad. Creemos interesante el uso de
     reconocimiento facial para funcionalidades de usuario, aunque pueda ser
     polémica es algo nuevo que aporta valor a la aplicación.
    \item La escena de RA esta implementada en Unity por lo que podemos
     exportarla a otras plataformas como iOS, consiguiendo con esto que la
     aplicación se abra a más usuarios y estos consigan un mayor crecimiento y
     enriquecimiento de la aplicación.
    \item Facilitar la compra de entradas desde la aplicación para que los
     usuarios puedan realizar la mayoría de las acciones, necesarias para ir al
     cine, redireccionando a la página web de venta de los cines a los que se
     quieren acudir para ver la película.
    \item Publicar la aplicación a una plataforma de distribución digital de
    aplicaciones como Google Play, para que los usuarios tengan acceso a
    instalarla en sus dispositivos.
    \item Automatizar con web scraping la obtención de datos de las fuentes de
     información de películas para que la aplicación siempre tenga la
     información actual.
    \item Utilizar distintos idiomas para el contenido de las películas en la
     base de datos del servidor usando una columna que indique el idioma de la
     información.
    \item Una de las cosas más importantes que falta por realizar es una
     evaluación con usuarios que nos proporcionen retroalimentación de la
     aplicación de la cual obtendremos una visión más amplia sobre si satisface
     la necesidad al problema que tratamos en el proyecto o si hay que realizar
     cambios. 
\end{enumerate}