% +--------------------------------------------------------------------+
% | Sample Chapter
% |
% | This file provides examples of how to
% | - insert a figure with a caption
% | - construct a table with a caption
% | - create subsections within the chapter
% | - insert a reference to a Figure or Table
% | - make a citation
% +--------------------------------------------------------------------+

\cleardoublepage

% +--------------------------------------------------------------------+
% | Replace "Chapter Title" below with the title of your chapter.  LaTeX
% | will automatically number the chapters.
% +--------------------------------------------------------------------+

\chapter{Introducción}
%\label{ch:chapter1}
\label{makereference}

% +--------------------------------------------------------------------+
% |To create cross-references to figures, tables and segments
% |of text, LaTeX provides the following commands:
% |   \label{marker}
% |   \ref{marker}
% |   \pageref{marker}
% | where {marker} is a unique identifier.
% |
% | In the line above, we use \label{figure1} to mark a location
% | we wish to refer to later.  LATEX replaces \ref by the number of
% | the chapter, section, subsection, figure, or table after which the
% | corresponding \label command was issued. \pageref prints the page
% | number of the page where the \label command occurred.
% |
% +--------------------------------------------------------------------+


% +--------------------------------------------------------------------+
% | Replace \section headings below with the title of your
% | subsections.  LaTeX will automatically number the subsections 1.1,
% | 1.2, 1.3, etc.
% +--------------------------------------------------------------------+

\section{Antecedentes}
\label{makereference1.1}

\section{Objetivos}
\label{makereference1.2}
\begin{itemize}  
    \item Estudiar las tecnologías de realidad aumentada actuales e implementarlas.
    \item Analizar sistemas de recomendación, así como su uso dentro del proyecto.
    \item Desarrollar una aplicación para dispositivos móviles que aplique las anteriores tecnologías en un proyecto real.
    \item Definir unos casos de uso reales que aporten valor a la aplicación.
    \item Administrar y configurar servicios web que provisionen de información a la aplicación.
    \item Desplegar un servidor y codificarlo de forma que su modificación sea lo más sencilla posible.
    \item Coordinar el trabajo en grupo y la gestión de los cambios mediante el uso de prácticas de las metodologías ágiles.
    \item Investigar la mejor forma de mejorar la experiencia del usuario de nuestra aplicación.
    \item Aplicar los conocimientos y disciplinas aprendidos de la Ingeniería del Software.
\end{itemize}

\section{Plan de trabajo}
\label{makereference1.3}



