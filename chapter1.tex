% +--------------------------------------------------------------------+
% | Sample Chapter
% |
% | This file provides examples of how to
% | - insert a figure with a caption
% | - construct a table with a caption
% | - create subsections within the chapter
% | - insert a reference to a Figure or Table
% | - make a citation
% +--------------------------------------------------------------------+

\cleardoublepage

% +--------------------------------------------------------------------+
% | Replace "Chapter Title" below with the title of your chapter.  LaTeX
% | will automatically number the chapters.
% +--------------------------------------------------------------------+

\chapter{Introducción}
%\label{ch:chapter1}
\label{makereference}

\begin{flushleft}
    En este capítulo introduciremos la idea de nuestro proyecto, de dónde surge, qué objetivos queremos lograr y cómo nos hemos organizado a la hora de trabajar.
    Con este proyecto queremos solventar una serie de situaciones en las que una persona
    tiene ganas de ver cierta película, ya sea porque se ha encontrado el cartel de la nueva película
    de su saga favorita anunciada en la calle y no sabe con quién ir, o porque quizás un amigo le ha sugerido que le acompañe
    a ver una película de la que no saben nada por el mero hecho de ir al cine. 
    Decidimos ponernos manos a la obra y desarrollar una aplicación móvil que permitiera a sus usuarios actuar en distintas situaciones
    para que con el simple uso de la cámara de su smartphone, e incluso sin ella, pudieran de forma
    fácil e interactiva, organizarse para ir al cine o a ver la película a casa de otra persona. Además, podrían apoyarse en la 
    información que la aplicación les proporciona sobre las distintas películas para decidir si les es interesante ir a verla, sobre todo 
    si es para ir al cine ya que el usuario podría ahorrarse el gasto de acabar viendo una película que no le gusta por no poder saber si se 
    ajusta a sus gustos. Además, tener una red de contactos de personas interesadas en ver películas que a ti también te gusten y tener la posibilidad
    de quedar en un lugar, fecha y hora determinados refuerzan el atractivo de nuestra aplicación en el dominio del ocio.
\end{flushleft}
% +--------------------------------------------------------------------+
% |To create cross-references to figures, tables and segments
% |of text, LaTeX provides the following commands:
% |   \label{marker}
% |   \ref{marker}
% |   \pageref{marker}
% | where {marker} is a unique identifier.
% |
% | In the line above, we use \label{figure1} to mark a location
% | we wish to refer to later.  LATEX replaces \ref by the number of
% | the chapter, section, subsection, figure, or table after which the
% | corresponding \label command was issued. \pageref prints the page
% | number of the page where the \label command occurred.
% |
% +--------------------------------------------------------------------+


% +--------------------------------------------------------------------+
% | Replace \section headings below with the title of your
% | subsections.  LaTeX will automatically number the subsections 1.1,
% | 1.2, 1.3, etc.
% +--------------------------------------------------------------------+

\section{Antecedentes}
\label{makereference1.1}
\begin{flushleft}
    Para entender de donde surge nuestra idea tenemos que remontarnos al momento en el que se propuso este proyecto.
    El principal motivo es la mejora de un Trabajo de Fin de Grado anterior
    al nuestro, con el mismo título (\textbf{*referenciar al anterior TFG*}), cuyo enfoque y finalidad era el mismo. Nuestro objetivo principal es 
    ampliar su funcionalidad diferenciándonos, sobre todo, en las tecnologías usadas y 
    en los casos de uso representados.
    En nuestra aplicación no solo nos centramos en ir al cine y reconocer películas que estuvieran en la cartelera
    para que nos recomiende la que más se ajuste a nuestros gustos. Decidimos que el usuario podría crear planes con
    una película que le interesase ver y que sus amigos u otros usuarios pudieran unirse para ir todos juntos a verla.
    También nos permitirá reconocer imágenes de caras de usuarios para añadirlos como amigos o que nos recomiende en tiempo real
    que películas podemos ir a ver con éstos.
\end{flushleft}
\section{Objetivos}
\label{makereference1.2}
Nuestro objetivo principal es crear una aplicación completa que funcione en un entorno real, con una comunicación cliente servidor que además 
incluya Realidad Aumentada para aportar un valor añadido a su funcionalidad.
También tenemos una lista de subobjetivos a cumplir:
\begin{itemize}  
    \item Estudiar las tecnologías de realidad aumentada actuales.
    \item Analizar sistemas de recomendación, así como su uso dentro del proyecto.
    \item Diseñar una aplicación para dispositivos móviles que aplique las anteriores tecnologías en un proyecto real.
    \item Definir casos de uso reales que aporten valor a la aplicación.
    \item Administrar y configurar servicios web que provisionen de información a la aplicación.
    \item Desplegar un servidor e implementar sus servicios de forma que su modificación sea lo más sencilla posible.
    \item Coordinar el trabajo en grupo y la gestión de los cambios mediante el uso de prácticas de las metodologías ágiles.
    \item Investigar la mejor forma de mejorar la experiencia de usuario de nuestra aplicación.
\end{itemize}

\section{Plan de trabajo}
\begin{flushleft}
    Para comenzar con nuestro proyecto analizaremos el estado del arte, en el Capítulo 2, en el que estudiaremos los conceptos
     básicos de la Realidad Aumentada y buscaremos las tecnologías actuales a la fecha del proyecto
     que funcionen en dispositivos móviles. También realizaremos en este capítulo un análisis de aquellas aplicaciones que se asemejen
     a nuestra idea, junto con un estudio de las distintas fuentes de información de las que podemos obtener datos y una breve introducción
    a los sistemas de recomendación que podríamos usar.
\end{flushleft}

\begin{flushleft}     
    Después realizaremos las tareas de diseño, descritas en el Capítulo 3. Primero,
    desarrollaremos prototipos usando tecnologías de Realidad Aumentada en distintas plataformas y crearemos otros prototipos para nuestra capa de datos. 
    A continuación, nos plantearemos los distintos
    escenarios en los que sería usada nuestra aplicación, posteriormente realizaremos una lista de las funcionalidades 
    más importantes de nuestra aplicación y diseñaremos las interfaces de usuario que acompañaran a dichas funcionalidades. 
\end{flushleft}

\begin{flushleft}    
    El siguiente paso será la implementación del software, descrito en el Capítulo 4, para el cual aplicaremos
     algunas prácticas de la metodología SCRUM. El desarrollo se realizará
     con iteraciones cercanas a las dos semanas, comenzará con la selección de
     objetivos más prioritarios y terminará con las reuniones de revisión junto
     con los directores del Trabajo de Fin de Grado. Tras cada iteración, los miembros del equipo 
     realizaremos una reunión de retrospectiva para identificar en qué hemos fallado durante esta iteración y
     cómo podemos mejorar para la siguiente.
\end{flushleft}

\begin{flushleft}
    Más tarde, pasaremos a estudiar distintas formas de implementar un sistema
     de recomendación con el fin de elegir el que mejor se adapte a nuestro proyecto. Tras
     decidirnos comenzaremos a implementarlo y a probarlo.
\end{flushleft}

\begin{flushleft}
Finalmente hablaremos en el capítulo 5 de las conclusiones que hemos obtenido tras realizar el proyecto y 
de las contribuciones de cada uno de los integrantes del equipo al proyecto en el capítulo 6.
\end{flushleft}
\label{makereference1.3}



