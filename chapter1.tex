% +--------------------------------------------------------------------+
% | Sample Chapter
% |
% | This file provides examples of how to
% | - insert a figure with a caption
% | - construct a table with a caption
% | - create subsections within the chapter
% | - insert a reference to a Figure or Table
% | - make a citation
% +--------------------------------------------------------------------+

\cleardoublepage

% +--------------------------------------------------------------------+
% | Replace "Chapter Title" below with the title of your chapter.  LaTeX
% | will automatically number the chapters.
% +--------------------------------------------------------------------+

\chapter{Introducción}
%\label{ch:chapter1}
\label{makereference}


    En este capítulo introduciremos la idea de nuestro proyecto, de dónde surge, qué objetivos queremos lograr y cómo nos hemos organizado a la hora de trabajar.
    Con este proyecto queremos desarrollar una aplicación para solventar una serie de situaciones en las que una persona
    tiene ganas de ver cierta película, ya sea porque se ha encontrado el cartel de la nueva película
    de su saga favorita anunciada en la calle y no sabe con quién ir, o porque quizás un amigo le ha sugerido que le acompañe
    a ver una película de la que no saben nada por el mero hecho de ir al cine. 
    Decidimos ponernos manos a la obra y desarrollar una aplicación móvil que permitiera a sus usuarios actuar en distintas situaciones
    para que con el simple uso de la cámara de su smartphone, e incluso sin ella, pudieran de forma
    fácil e interactiva, organizarse para ir al cine o a ver la película a casa de otra persona. Además, podrían apoyarse en la 
    información que la aplicación les proporciona sobre las distintas películas para decidir si les es interesante ir a verla, sobre todo 
    si es para ir al cine ya que el usuario podría ahorrarse el gasto de acabar viendo una película que no le gusta por no poder saber si se 
    ajusta a sus gustos. Además, tener una red de contactos de personas interesadas en ver películas que a ti también te gusten y tener la posibilidad
    de quedar en un lugar, fecha y hora determinados refuerzan el atractivo de nuestra aplicación en el dominio del ocio.
    Nuestro proyecto ofrecerá a los usuarios explicaciones sobre el producto que se esté reconociendo mediante la realidad aumentada, es decir, al usuario se le 
    recomendarán productos relevantes de la aplicación según sus gustos, también se le mostrará información sobre los objetos que se reconocen.

% +--------------------------------------------------------------------+
% |To create cross-references to figures, tables and segments
% |of text, LaTeX provides the following commands:
% |   \label{marker}
% |   \ref{marker}
% |   \pageref{marker}
% | where {marker} is a unique identifier.
% |
% | In the line above, we use \label{figure1} to mark a location
% | we wish to refer to later.  LATEX replaces \ref by the number of
% | the chapter, section, subsection, figure, or table after which the
% | corresponding \label command was issued. \pageref prints the page
% | number of the page where the \label command occurred.
% |
% +--------------------------------------------------------------------+


% +--------------------------------------------------------------------+
% | Replace \section headings below with the title of your
% | subsections.  LaTeX will automatically number the subsections 1.1,
% | 1.2, 1.3, etc.
% +--------------------------------------------------------------------+

\section{Antecedentes}
\label{makereference1.1}

Para entender de donde surge nuestra idea tenemos que remontarnos al momento en el que se propuso este proyecto.
El principal motivo es la mejora de un Trabajo de Fin de Grado anterior
 al nuestro\cite{TFGRA16}, cuyo enfoque y finalidad era el mismo. Nuestro objetivo principal es 
 ampliar su funcionalidad diferenciándonos, sobre todo, en las tecnologías usadas y 
 en los casos de uso representados.

 En nuestra aplicación no solo nos centramos en ir al cine y reconocer películas que estuvieran en la cartelera
 para que nos recomiende la que más se ajuste a nuestros gustos. Decidimos que el usuario podría crear planes con
 una película que le interesase ver y que sus amigos u otros usuarios pudieran unirse para ir todos juntos a verla.
 También añadiremos más interacción sobre las escenas de realidad aumentada, permitiéndonos ver la información sobre la 
 película reconocida, ver trailers o guardarla en la lista de favoritos. Además, con la realidad aumentada,
 se podrán realizar reconocimientos faciales sobre los usuarios para añadirlos como amigos, para después recomendarnos en tiempo real
 que películas podemos ir a ver con éstos.

\section{Objetivos}
\label{makereference1.2}
Nuestro objetivo principal es crear una aplicación que ayude a los usuarios a
realizar planes con sus amigos para ir al cine a ver películas. Además,
mostrará qué películas pueden interesar al usuario en base a sus gustos.
También se usarán tecnologías de realidad aumentada que ayuden al usuario a usar
la aplicación.
La lista de subobjetivos es la siguiente:
\begin{itemize}  
    \item Estudiar las tecnologías de realidad aumentada actuales.
    \item Analizar sistemas de recomendación, así como su uso dentro del
     proyecto.
    \item Definir casos de uso reales que aporten valor a la aplicación.
    \item Diseñar una aplicación para dispositivos móviles que aplique las
     anteriores tecnologías.
    \item Diseñar y aplicar una arquitectura que cumpla las necesidades del
     proyecto.
\end{itemize}

\section{Plan de trabajo}
\label{makereference1.3}

Para comenzar con nuestro proyecto analizaremos el estado del arte, en el
 \autoref{makereference2}, en el que estudiaremos los conceptos básicos de la
 Realidad Aumentada y buscaremos las tecnologías actuales a la fecha del proyecto
 que funcionen en dispositivos móviles. También realizaremos un estudio de las distintas
 fuentes de información de las que podemos obtener datos y un estudio de
 técnicas de recomendación.

Después realizaremos las tareas de diseño, descritas en el
 \autoref{makereference3}. Primero, un análisis de aquellas aplicaciones que se
 asemejen a nuestra idea. A continuación, nos plantearemos los distintos
 escenarios en los que sería usada nuestra aplicación. Posteriormente
 realizaremos una lista de las funcionalidades más importantes de nuestra
 aplicación y diseñaremos las interfaces de usuario que acompañarán a dichas
 funcionalidades. 
   
El siguiente paso, descrito en el \autoref{makereference4}, comenzará con
 el desarrollo de una serie de prototipos con distintas tecnologías que nos permitan
 decidir cuales utilizar en la implementación de la aplicación final. Después diseñaremos la
 arquitectura e implementaremos la aplicación, para lo cual aplicaremos algunas
 prácticas de la metodología SCRUM. El desarrollo se realizará con iteraciones
 cercanas a las dos semanas. Comenzará con la selección de objetivos más
 prioritarios y terminará con las reuniones de revisión junto con los directores
 del Trabajo de Fin de Grado. Tras cada iteración, los miembros del equipo
 realizaremos una reunión de retrospectiva para identificar en qué hemos fallado
 durante esta iteración y cómo podemos mejorar para la siguiente.

Finalmente en el \autoref{makereference5}, hablaremos de las conclusiones que
 hemos obtenido tras realizar el proyecto, seguido del trabajo futuro que se podría realizar sobre nuestra
 aplicación para conseguir futuras mejoras en el \autoref{makereference5.2}.

\section{Introduction}

In this chapter we are going to introduce the idea of our project, where it comes from,
 what objectives we want to achieve and how we have organized ourselves to work 
together. With this project we want to develop an application to solve a series 
of situations in which a person wants to watch a certain movie, either because he 
has encountered the poster of a new movie of it$'$s favorite saga announced on the 
street and doesn$'$t know who to go with, or because maybe a friend of him has 
suggested on going to see a movie together they don$'$t know about 
by the mere fact of going to the cinema. We decided to get to work and develop a 
mobile application that would allow it$'$s users to act in different situations so 
that with the simple use of their smarthpone$'$s camera, or even without it, 
they could get organized to go to the cinema or to watch the movie in another 
person$'$s house in an easy and interactive way. Also, they could rely on 
the information that the application provides them about the different movies so they can decide 
whether to watch it or not, especially if it is for going to the cinema since the user could save himself 
the expense of ending up watching a movie thath he doesn$'$t like because he can$'$t know if it fits 
his tastes. Furthermore, having a network of contacts of people interested in going to watch films that
you like too and having the possibility of meeting on a certain place, date and hour will reinforce the attractive
of our application in leisure$'$s domain.

\section{Background}

To understand where our idea comes from we have to go back to the moment in which this project was proposed.
The main reason is the improvement of a Final Degree Project previous to ours \cite{TFGRA16}, which approach and purpose was the same. Our main 
objective is to extend it$'$s functionality differentiating us, especially, in the technologies used and in the represented use cases.

In our application we don$'$t just focus on going to the cinema and recognizing films that were 
on the billboard so that it can recommend us the movie that fits our tastes better. We decided that an user
could create plans with a film he is interested in watching and that his friends or other users could join so they can
all go together. We will also add more interaction to the augmented reality scenes, allowing us to see the information about
the film we are recognizing, watch trailers or save it to our favorites list. Furthermore, with augmented
reality, it will be possible to realize facial recognitions of users so we can add them as friends, so it can afterwards recommends us
films that we can go watch with them in real time.

\section{Objectives}

Our main objective is to create an application that will help the users to carry out plans with their friends to go
to the cinema to watch movies. It will also show what films are interesting for the user based on his tastes. We will also use
augmented reality technologies that will help the user to use the application. The list of subobjectives is the next one:

\begin{itemize}
    \item Study the current augmented reality technologies.
    \item Analize recommendation systems, as well as it$'$s use inside the project.
    \item Define real uses cases that will add value to the application.
    \item Design an application for mobile devices that will apply the technologies described before.
    \item Design and apply an architecture that meets the needs of the project.
\end{itemize}

\section{Workplan}

To start with our project we will analize the state of art, in Chapter 2, in which we will
study the basic concepts of augmented reality and we will search the current technologies to the project$'$s date that
work in mobile devices. We will also perform a study of the different information sources from which we can obtain data and we will
also perform a study about recommendation techniques.

Afterwards we will perform design tasks, described in Chapter 3. First, an analysis of those 
applications which are similar to our idea. Then, we will consider the different scenarios in which
our application will be used. Later we will do a list of the most important functionalities of our application 
and design the user interfaces that will accompany these functionalities.

The next step, described in Chapter 4, will start with the development of a series of prototypes with different
technologies that will allow us to decide which ones to use in the implementation of our final application. After we 
will design the architectura and implement the application, for it we will apply some practices of the SCRUM methodology. 
The development will be carried out with iterations close to the two weeks of duration. It will start with the selection 
of the most priority objectives and will end with review meetings alongside the Final Degree Project directors. After each iteration,
all of the team members will attend a retrospective meeting to identify in what we have failed during this iteration and 
how we could improve for the next one. 

Finally in Chapter 5, we will speak about the conclusions we have obtained from implementing this project, followed by the future work that
could be done in our application tu achieve future improvements in Section 5.2.
