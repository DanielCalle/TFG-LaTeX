% +--------------------------------------------------------------------+
% | Sample Chapter
% |
% | This file provides examples of how to
% | - insert a figure with a caption
% | - construct a table with a caption
% | - create subsections within the chapter
% | - insert a reference to a Figure or Table
% | - make a citation
% +--------------------------------------------------------------------+

\cleardoublepage

% +--------------------------------------------------------------------+
% | Replace "Chapter Title" below with the title of your chapter.  LaTeX
% | will automatically number the chapters.
% +--------------------------------------------------------------------+

\chapter{Introducción}
%\label{ch:chapter1}
\label{makereference}

% +--------------------------------------------------------------------+
% |To create cross-references to figures, tables and segments
% |of text, LaTeX provides the following commands:
% |   \label{marker}
% |   \ref{marker}
% |   \pageref{marker}
% | where {marker} is a unique identifier.
% |
% | In the line above, we use \label{figure1} to mark a location
% | we wish to refer to later.  LATEX replaces \ref by the number of
% | the chapter, section, subsection, figure, or table after which the
% | corresponding \label command was issued. \pageref prints the page
% | number of the page where the \label command occurred.
% |
% +--------------------------------------------------------------------+


% +--------------------------------------------------------------------+
% | Replace \section headings below with the title of your
% | subsections.  LaTeX will automatically number the subsections 1.1,
% | 1.2, 1.3, etc.
% +--------------------------------------------------------------------+

\section{Antecedentes}
\label{makereference1.1}
\begin{flushleft}
    Esta idea surge de la propuesta de mejora de un Trabajo de Fin de Grado anterior
    al nuestro cuyo enfoque y finalidad era el mismo. Nuestro objetivo principal es 
    ampliar su funcionalidad diferenciándonos, sobre todo, en las tecnologías usadas y 
    en los casos de uso representados.
    En nuestra aplicación hemos cambiado el enfoque desde la perspectiva del usuario, no 
    nos centramos únicamente en el acto de ir al cine y reconocer películas que estuvieran en la cartelera
    para que nuestra aplicación nos recomiende la que más se ajuste a nuestros gustos. Si no que, hemos 
    abogado por la creación de planes, en los cuales podemos incluir películas que hayamos visto o que queramos ver
    y pueden unirse otros usuarios de la aplicación, haciendo así que su funcionalidad sea mucho más interactiva y 
    abierta al uso en el día a día.
    En cuanto a la diferencia desde el punto de vista tecnológico, hemos decidido usar \textbf{Unity} y \textbf{Vuforia} para el reconocimiento
    de imágenes y el uso de la Realidad Aumentada para aportar valor a la experiencia de usuario. En cuanto a la parte backend, 
    hemos usado \textbf{Spring} como framework para nuestra \textbf{API REST}, ya que está siendo usado ampliamente a nivel empresarial actualmente.
\end{flushleft}
\newpage
\section{Objetivos}
\label{makereference1.2}
\begin{itemize}  
    \item Estudiar las tecnologías de realidad aumentada actuales e implementarlas.
    \item Analizar sistemas de recomendación, así como su uso dentro del proyecto.
    \item Desarrollar una aplicación para dispositivos móviles que aplique las anteriores tecnologías en un proyecto real.
    \item Definir casos de uso reales que aporten valor a la aplicación.
    \item Administrar y configurar servicios web que provisionen de información a la aplicación.
    \item Desplegar un servidor e implementar sus servicios de forma que su modificación sea lo más sencilla posible.
    \item Coordinar el trabajo en grupo y la gestión de los cambios mediante el uso de prácticas de las metodologías ágiles.
    \item Investigar la mejor forma de mejorar la experiencia de usuario de nuestra aplicación.
    \item Aplicar los conocimientos y disciplinas aprendidos de la Ingeniería del Software.
\end{itemize}

\newpage
\section{Plan de trabajo}
\begin{flushleft}
    Para comenzar analizaremos el estado del arte, en el que aprenderemos los conceptos
     básicos de la Realidad Aumentada y buscaremos las tecnologías actuales a las fechas del proyecto,
     que funcionen en dispositivos móviles.
\end{flushleft}

\begin{flushleft}     
    Después realizaremos las tareas de diseño en las que desarrollaremos prototipos usando
     tecnologías de Realidad Aumentada en distintas plataformas y crearemos otros prototipos para una \textbf{API REST}.
     También realizaremos una lista de las funcionalidades más importantes de nuestra aplicación
     y diseñaremos bocetos para la interfaz de usuario, que nos servirán como guía en
     la implementación.
\end{flushleft}

\begin{flushleft}    
    El siguiente paso será la implementación del software para el cual aplicaremos
     algunas prácticas de la metodología \textbf{SCRUM}. Nuestro proyecto estará dividido en tres repositorios, uno para la
     \textbf{API REST} y dos más, uno de ellos para la parte de \textbf{RA} que se integra como una
     escena en el segundo proyecto, realizado en \textbf{Android Studio}. El desarrollo se realizará
     con iteraciones cercanas a las dos semanas, comenzará con la selección de
     objetivos más prioritarios y terminará con las reuniones de revisión junto
     con los directores del Trabajo de Fin de Grado, posteriormente otra reunión de retrospectiva en
     la que participa únicamente el equipo, que nos servirá para realizar mejoras en la
     siguiente iteración.
\end{flushleft}

\begin{flushleft}
    Finalmente, pasaremos a estudiar distintas propuestas para el sistema
     de recomendación con el fin de elegir la que mejor se adapte a nuestro
     proyecto e implementarla como servicio para que la aplicación pueda
     utilizarlo.
\end{flushleft}

\label{makereference1.3}



