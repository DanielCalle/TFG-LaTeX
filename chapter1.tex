% +--------------------------------------------------------------------+
% | Sample Chapter
% |
% | This file provides examples of how to
% | - insert a figure with a caption
% | - construct a table with a caption
% | - create subsections within the chapter
% | - insert a reference to a Figure or Table
% | - make a citation
% +--------------------------------------------------------------------+

\cleardoublepage

% +--------------------------------------------------------------------+
% | Replace "Chapter Title" below with the title of your chapter.  LaTeX
% | will automatically number the chapters.
% +--------------------------------------------------------------------+

\chapter{Introducción}
%\label{ch:chapter1}
\label{makereference}


    En este capítulo introduciremos la idea de nuestro proyecto, de dónde surge, qué objetivos queremos lograr y cómo nos hemos organizado a la hora de trabajar.
    Con este proyecto queremos desarrollar una aplicación para solventar una serie de situaciones en las que una persona
    tiene ganas de ver cierta película, ya sea porque se ha encontrado el cartel de la nueva película
    de su saga favorita anunciada en la calle y no sabe con quién ir, o porque quizás un amigo le ha sugerido que le acompañe
    a ver una película de la que no saben nada por el mero hecho de ir al cine. 
    Decidimos ponernos manos a la obra y desarrollar una aplicación móvil que permitiera a sus usuarios actuar en distintas situaciones
    para que con el simple uso de la cámara de su smartphone, e incluso sin ella, pudieran de forma
    fácil e interactiva, organizarse para ir al cine o a ver la película a casa de otra persona. Además, podrían apoyarse en la 
    información que la aplicación les proporciona sobre las distintas películas para decidir si les es interesante ir a verla, sobre todo 
    si es para ir al cine ya que el usuario podría ahorrarse el gasto de acabar viendo una película que no le gusta por no poder saber si se 
    ajusta a sus gustos. Además, tener una red de contactos de personas interesadas en ver películas que a ti también te gusten y tener la posibilidad
    de quedar en un lugar, fecha y hora determinados refuerzan el atractivo de nuestra aplicación en el dominio del ocio.

% +--------------------------------------------------------------------+
% |To create cross-references to figures, tables and segments
% |of text, LaTeX provides the following commands:
% |   \label{marker}
% |   \ref{marker}
% |   \pageref{marker}
% | where {marker} is a unique identifier.
% |
% | In the line above, we use \label{figure1} to mark a location
% | we wish to refer to later.  LATEX replaces \ref by the number of
% | the chapter, section, subsection, figure, or table after which the
% | corresponding \label command was issued. \pageref prints the page
% | number of the page where the \label command occurred.
% |
% +--------------------------------------------------------------------+


% +--------------------------------------------------------------------+
% | Replace \section headings below with the title of your
% | subsections.  LaTeX will automatically number the subsections 1.1,
% | 1.2, 1.3, etc.
% +--------------------------------------------------------------------+

\section{Antecedentes}
\label{makereference1.1}

Para entender de donde surge nuestra idea tenemos que remontarnos al momento en el que se propuso este proyecto.
El principal motivo es la mejora de un Trabajo de Fin de Grado anterior
 al nuestro\cite{TFGRA16}, cuyo enfoque y finalidad era el mismo. Nuestro objetivo principal es 
 ampliar su funcionalidad diferenciándonos, sobre todo, en las tecnologías usadas y 
 en los casos de uso representados.

 En nuestra aplicación no solo nos centramos en ir al cine y reconocer películas que estuvieran en la cartelera
 para que nos recomiende la que más se ajuste a nuestros gustos. Decidimos que el usuario podría crear planes con
 una película que le interesase ver y que sus amigos u otros usuarios pudieran unirse para ir todos juntos a verla.
 También añadiremos más interacción sobre las escenas de realidad aumentada, permitiéndonos ver la información sobre la 
 película reconocida, ver trailers o guardarlo en la lista de favoritos. Además, con la realidad aumentada,
 se podrán realizar reconocimientos faciales sobre los usuarios para añadirlos como amigos, para después recomendarnos en tiempo real
 que películas podemos ir a ver con éstos.

\section{Objetivos}
\label{makereference1.2}
Nuestro objetivo principal es crear una aplicación que ayude a los usuarios a
realizar planes con sus amigos para ir al cine a ver películas. Además,
mostrará qué películas pueden interesar al usuario en base a sus gustos.
También se usarán tecnologías de Realidad Aumentada que ayuden al usuario a usar
la aplicación.
La lista de subobjetivos es la siguiente:
\begin{itemize}  
    \item Estudiar las tecnologías de realidad aumentada actuales.
    \item Analizar sistemas de recomendación, así como su uso dentro del
     proyecto.
    \item Definir casos de uso reales que aporten valor a la aplicación.
    \item Diseñar una aplicación para dispositivos móviles que aplique las
     anteriores tecnologías.
    \item Diseñar y aplicar una arquitectura que cumpla las necesidades del
     proyecto.
\end{itemize}

\section{Plan de trabajo}
\label{makereference1.3}

Para comenzar con nuestro proyecto analizaremos el estado del arte, en el
 \autoref{makereference2}, en el que estudiaremos los conceptos básicos de la
 Realidad Aumentada y buscaremos las tecnologías actuales a la fecha del proyecto
 que funcionen en dispositivos móviles. También realizaremos un estudio de las distintas
 fuentes de información de las que podemos obtener datos y un estudio de
 técnicas de recomendación.

Después realizaremos las tareas de diseño, descritas en el
 \autoref{makereference3}. Primero, un análisis de aquellas aplicaciones que se
 asemejen a nuestra idea. A continuación, nos plantearemos los distintos
 escenarios en los que sería usada nuestra aplicación. Posteriormente
 realizaremos una lista de las funcionalidades más importantes de nuestra
 aplicación y diseñaremos las interfaces de usuario que acompañarán a dichas
 funcionalidades. 
   
El siguiente paso, descrito en el \autoref{makereference4}, comenzará con
 el desarrollo de una serie de prototipos con distintas tecnologías que nos permitan
 decidir cuales utilizar en la implementación de la aplicación final. Después diseñaremos la
 arquitectura e implementaremos la aplicación, para lo cual aplicaremos algunas
 prácticas de la metodología SCRUM. El desarrollo se realizará con iteraciones
 cercanas a las dos semanas. Comenzará con la selección de objetivos más
 prioritarios y terminará con las reuniones de revisión junto con los directores
 del Trabajo de Fin de Grado. Tras cada iteración, los miembros del equipo
 realizaremos una reunión de retrospectiva para identificar en qué hemos fallado
 durante esta iteración y cómo podemos mejorar para la siguiente.

Finalmente en el \autoref{makereference5}, hablaremos de las conclusiones que
 hemos obtenido tras realizar el proyecto, seguido del trabajo futuro que se podría realizar sobre nuestra
 aplicación para conseguir futuras mejoras en el \autoref{makereference5.2}.
