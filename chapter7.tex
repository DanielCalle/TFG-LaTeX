% +--------------------------------------------------------------------+
% | Sample Chapter 7
% +--------------------------------------------------------------------+

\cleardoublepage

% +--------------------------------------------------------------------+
% | Replace "This is Chapter 7" below with the title of your chapter.
% | LaTeX will automatically number the chapters.
% +--------------------------------------------------------------------+

\chapter{Contribución al proyecto}
\label{makereference7}

\section{Diego Acuña Berger}
\label{makereference7.1}
\begin{itemize}
    \item Funcionalidad para guardar películas mediante realidad aumentada.
    \item Diseño de las interfaces de usuario de los planes y películas guardadas.  
    \item Estudio y creación del prototipo de servidor en \textbf{Spring}.
    \item Revisión ortográfica de la memoria.
    \item Obtención de imágenes mediante \textbf{Picasso} a la aplicación.
    \item Disposición y diseño de botones que aparecen tras reconocer una película con realidad aumentada.
    \item Redacción de las distintas interfaces de usuario de la aplicación.
    \item Redacción de los stakeholders.
    \item Diseños de vistas de la aplicación con \textbf{MockFlow}.
\end{itemize}
\section{Daniel Calle Sánchez}
\label{makereference7.2}
\begin{itemize}  
    \item Creación de prototipo con \textbf{ViroReact} y \textbf{ARCore}.
    \item Identificación de requisitos funcionales de la aplicación.
    \item Búsqueda y configuración de herramientas que mejoren el rendimiento del equipo.
    \item Diseño de la arquitectura.
    \item Diseños de vistas de la aplicación con \textbf{MockFlow}.
    \item Configuración de \textbf{Heroku}.
    \item Implementación de servicios en \textbf{Spring}.
    \item Reconocimiento de usuarios con \textbf{Vuforia} en \textbf{Unity.}
    \item Desarrollo de vistas para planes en \textbf{Android}.
\end{itemize}

\section{Carlos Gómez Cereceda}
\label{makereference7.3}
\begin{itemize}  
    \item Creación de prototipo con ARCore en Android.
    \item Creación de prototipo con Vuforia en Unity.
    \item Creación de prototipo con Vuforia en Unity exportado a Android.
    \item Configuración del proyecto para poder tener la parte de Vuforia separada de la de Android.
    \item Comunicación entre la parte de Vuforia en Unity y la de Android.
    \item Configuración del proyecto para que se pueda comunicar con el Cloud de Vuforia.
    \item Creación de las primeras vistas prototipo del proyecto, tales como: Mostrar vídeos en Realidad Aumentada, botones y películas.
    \item Comunicación en Vuforia con el servidor.
    \item Creación de la vista en Unity que se muestra al identificar una película mediante Realidad Aumentada, además de su funcionalidad y comunicación con Android. Ejemplo: mostrar tráiler, puntuación, información…
    \item Creación de la vista en Unity que se muestra al identificar a un usuario no identificado como amigo. Así como la funcionalidad en realidad Aumentada y Android para agregar a dicho usuario como amigo.
    \item Creación de la vista en Unity que se muestra al identificar a un amigo en donde se muestran los tres planes recomendados y la información visual de cuánto se prevé que les guste ese plan a los usuarios dentro del plan.
    \item Creación en Android de métodos de comunicación con el servidor para acciones como añadir amigos, borrar plan, añadir amigo, obtener amistad…
    \item Creación en el servidor de la primera versión de usuarios y su relación con películas.
\end{itemize}

\section{Zihao Hong}
\label{makereference7.4}
\begin{itemize}  
    \item Creación de prototipo con Vuforia en Android.
    \item Creación de prototipo de la arquitectura de Microservicios.
    \item Evaluación y pruebas para establecer el proyecto de Unity.
    \item Evaluación del uso del Docker ofrecido por la universidad.
    \item Diseño de arquitectura del proyecto de android y de las vistas.
    \item Configuración del proyecto del servidor mediante maven y perfiles para el despliegue en un entorno Linux.
    \item Estudio de sistemas de recomendación y su implementación en el servidor.
    \item Revisor de código.
    \item Establecimiento del uso de patrones de diseño. 
\end{itemize}