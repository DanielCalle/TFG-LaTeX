% +--------------------------------------------------------------------+
% | Sample Chapter 7
% +--------------------------------------------------------------------+

\cleardoublepage

% +--------------------------------------------------------------------+
% | Replace "This is Chapter 7" below with the title of your chapter.
% | LaTeX will automatically number the chapters.
% +--------------------------------------------------------------------+

\chapter{Contribución al proyecto}
\label{makereference7}
En este capítulo especificaremos las aportaciones de cada integrante del grupo,
 las cuales se ordenarán por las fases del proyecto.

\section{Diego Acuña Berger}
\label{makereference7.1}
    \subsection{Estado del arte}
    \label{makereference7.1.1}
        \begin{itemize}
            \item Investigación de las distintas fuentes de información para la obtención de los datos referentes para las películas, valorando cuál sería la que mejor información aportaba para el proyecto.
        \end{itemize}
    \subsection{Diseño de la aplicación}
    \label{makereference7.1.2}
        \begin{itemize}
            \item Diseños de vistas de la aplicación con MockFlow. Es decir, el diseño de los primeros bocetos de las interfaces que terminaron por ser lo que hemos diseñado en el \autoref{makereference3}.
            \item Diseño de las interfaces de usuario de los planes y películas
            guardadas. Tras varias pruebas y diseños fallidos diseñé la forma de mostrar los planes y películas basándome en aplicaciones del ámbito del cine o de las series, como puede ser TV Time \cite{tvtime}
            \item Diseño de la vista que aparece tras guardar una película mediante la realidad aumentada, con la que se confirma que película se ha guardado.
        \end{itemize}
    \subsection{Implementación}
    \label{makereference7.1.3}
        \begin{itemize}
            \item Estudio y creación del prototipo de servidor en Spring. Búsqueda de tutoriales y la codificación de un pequeño servidor en local usando el framework de Spring, tras el prototipo contribuí a su ampliación y modificación para que se ajustase correctamente a lo que necesitaba nuestra aplicación.
            \item Obtención de imágenes mediante Picasso a la aplicación. 
            \item Funcionalidad para guardar películas mediante realidad aumentada, esto implica la comunicación con Android y Unity para saber la relación que tiene la película que se ha reconocido desde el cloud, con la película que se guarda en el servidor y otorgar feedback al usuario de su correcto funcionamiento.
            \item Disposición y diseño de botones que aparecen tras reconocer una
            película con realidad aumentada. Los botones que mostrábamos al principio no tenían mucho sentido con sus iconos y ocupaban mucho espacio, opté por incluir un botón inicial que al pulsarse se desplegase y mostrase más botones por encima del mismo, en los cuales puedes acceder a la información de la película o guardarla.
            \item Implementación de funciones del servidor tales como unirse a un plan, salirse de un plan y buscar los planes de un usuario concreto. Cuando me dedicaba a implementar ciertas vistas de la aplicación de Android, como mostrar los planes o las películas guardadas, me vi en la necesidad de incluir funciones en el servidor para obtener los datos que necesitaba, las cuales no nos habíamos planteado ninguno hasta el momento en el que fueron necesarias. 
        \end{itemize}
    \subsection{Memoria}
    \label{makereference7.1.3}
        \begin{itemize}
            \item Redacción del resumen tanto en español como en inglés.
            \item Redacción de la introducción a nuestro proyecto así como de los antecedentes.
            \item Redacción de los escenarios en los que se usaría nuestra aplicación.
            \item Redacción de los requisitos funcionales de la aplicación.
            \item Redacción de las distintas interfaces de usuario de la aplicación.
            \item Redacción del prototipo del servidor que implementé con Spring.
            \item Redacción de cómo está implementado nuestro servidor, es decir, qué entidades poseemos, que servicios de aplicación, controladores y repositorios.
            \item Revisión de calidad de la memoria. Me he ocupado de revisar las posibles faltas ortográficas o fallos de estilo al usar LaTeX.
            \item Redacción de las dificultades de trabajo así como de las soluciones que les dimos durante el desarrollo del proyecto.
        \end{itemize}

\section{Daniel Calle Sánchez}
\label{makereference7.2}
    \subsection{Estado del arte}
    \label{makereference7.2.1}
        \begin{itemize}
            \item Investigación de las tecnologías de Realidad Aumentada ViroReact y
            Expo AR. Con el fin de decidir el entorno en el que finalmente se implementará la aplicación.
        \end{itemize}
    \subsection{Diseño de la aplicación}
    \label{makereference7.2.2}
        \begin{itemize}
            \item Colaboración en la identificación de requisitos funcionales de la aplicación.
            \item Creación de los segundos bocetos para las interfaces en
             MockFlow, definiendo un recorrido en el uso de la aplicación.
        \end{itemize}
    \subsection{Implementación}
    \label{makereference7.2.3}
        \begin{itemize}
            \item Creación de un \href{https://github.com/DanielCalle/react-native-demo}{prototipo con ViroReact y React Native}
             siguiendo los ejemplos de la documentación oficial de Viro Media\cite{viroreact},
             incluye un menú desde el que se accede a dos escenas de RA, una
             que muestra texto y otra que muestra al personaje de una película
             saliendo del poster de la película cuando el
             dispositivo lo reconoce. También se usa la librería llamada NativeBase que
             muestra distinta apariencia en las interfaces si es iOS o Android.
             Se solucionó problemas de dependencias al usar las distintas librerías.
            \item Creación de un \href{https://github.com/DanielCalle/DemoARCore}{prototipo con ARCore usando Kotlin} como lenguaje
            en la plataforma de Android, siguiendo los ejemplos proporcionados por
            la documentación de Google ARCore\cite{arcore}.
            \item Gestión de los repositorios utilizados para el proyecto en 
            GitHub.
            \item Diseño de la arquitectura que ha permitido la comunicación
             entre servidor y cliente.
            \item Configuración de los recursos y la aplicación en Heroku para
             el despliegue automático. 
            \item Implementación de los primeros servicios para planes y amigos
             que contenían todas las operaciones CRUD en Spring.
            \item Cambios en los servicios de usuarios y películas para añadir
             nuevas operaciones y datos.
            \item Implementación de controles a los datos de entrada en los servicios del Spring.
            \item Uso de patrón Transfer para los datos de entrada y salida en Spring.
            \item Uso de los patrones Factoría y Singleton en Spring.
            \item Cambios en los métodos HTTP de los servicios de Spring para
            usar los correctos que describen los estándares de API REST.
            \item Implementación del reconocimiento de usuarios con Vuforia en Unity
             para que al detectar que la imagen es la de un usuario mostrar los
             paneles correspondientes y si es una película mostrar los anteriores
             ya existentes.
            \item Cambios en la lista de planes en la interfaz de Android para
             mostrar fecha, imágenes de usuarios.
            \item Guiar las reuniones de retrospectiva utilizando algunas dinámicas
             como la técnica del barco velero para recolectar datos del equipo y
             generar ideas. Después de la anterior, aplicamos la lista abierta de
             acciones para decidir qué hacer con el fin de mejorar en la
             siguiente iteración.
            \item Búsqueda y configuración de herramientas que mejoren el
             rendimiento de trabajo en equipo como Slack, Trello.
        \end{itemize}
    \subsection{Memoria}
    \label{makereference7.2.3}
        \begin{itemize}
            \item Definir estructura de la memoria.
            \item Redacción de \autoref{makereference5} Conclusiones.
            \item Redacción de \autoref{makereference5.2} Trabajo futuro.
            \item Referencias a la bibliografía.
            \item Revisión de la memoria.
        \end{itemize}

        \section{Carlos Gómez Cereceda}
        \label{makereference7.3}
            \subsection{Estado del arte}
            \label{makereference7.3.1}
                \begin{itemize}
                    \item Investigación de Vuforia en Unity y su exportación a Android, con el fin de decidir el entorno en el que 
                    finalmente se implementaría la aplicación. 
                    \item Estudio del funcionamiento del Cloud de Vuforia, con el objetivo de decidir su viabilidad en la incorporación 
                    al proyecto y si con la licencia gratuita ofrecida por Vuforia era suficiente. 
                    \item Investigación de ARCore con el objetivo de decidir si las funcionalidades ofrecidas por esta tecnología eran las que necesitábamos.
                \end{itemize}
            \subsection{Diseño de la aplicación}
            \label{makereference7.3.2}
                \begin{itemize}
                    \item Diseño de las interfaces mostradas en Realidad Aumentada cuando se reconoce a un usuario o a una película. 
                    \item Primeras interfaces desarrolladas con Realidad Aumentada en las que se mostraban vídeos. En un primer momento, 
                    pensamos que sería buena idea mostrar los tráileres de las películas en Realidad Aumentada, aunque finalmente se desechó la idea porque saturaba la interfaz.
                    \item Participación en la identificación de escenarios y creando los primeros bocetos.
                \end{itemize}
            \subsection{Implementación}
            \label{makereference7.3.3}
                \begin{itemize}  
                    \item Estudio y creación de una aplicación desarrollada enteramente en Unity con Vuforia. Puesto que ninguno de los integrantes 
                    del equipo habíamos tenido ninguna experiencia con Unity ni con Vuforia, mediante tutoriales y documentación desarrollé un prototipo 
                    básico en el que se detectaba una imagen objetivo y se situaba encima de ésta un modelo 3D.
                    \item Una vez desarrollado el prototipo anterior, investigué como de factible era realizar el resto de la aplicación en Unity. Realicé 
                    alguna interfaz y funcionalidades básicas, finalmente decidimos que no era viable ya que Unity no proporciona demasiadas facilidades ya 
                    que no es para lo que está diseñado. Por lo investigué cual era la forma de exportar el proyecto de Unity a uno en Android, creando un 
                    prototipo simple en Android que al pulsar un botón llevase a Unity.
                    \item Finalmente, al prototipo anterior le incluí el Cloud de Vuforia para no sobrecargar la aplicación con imágenes.
                    \item Estudio y creación de un prototipo de ARCore en Android. Con la guía oficial de ARCore realicé un prototipo de ARCore, sin embargo, 
                    la comparación Vuforia era muy grande, éste último era mucho más sencillo de aprender e intuitivo por lo que finalmente no se realizó el proyecto con ARCore.
                    \item Una vez teníamos el proyecto en Android con la parte de Realidad Aumentada en Unity con Vuforia investigué como enviar información 
                    de una parte a otra, ya que en cierto modo son partes distintas de la aplicación.
                    \item Desarrollé las primeras versiones de la parte de Realidad Aumentada en las que se mostraban vídeos con los tráileres de las películas y texto con la sinopsis.
                    \item Comunicación en Vuforia con el servidor, aunque finalmente decidimos no usar este método y que todas las llamadas al servidor se realizasen desde Android.
                    \item Diseñé e implementé la mecánica que permite distinguir si el objeto al que se está enfocando es una película o un usuario sin necesidad de 
                    tener dos tipos de Realidad Aumentada. Al igual que la mecánica anterior, una vez que se sabe que es un usuario, desarrollé la mecánica para saber si es un amigo o no.
                    \item Creación con ayuda de los ejemplos de Vuforia del escáner que proporciona al usuario el feedback de que está escaneando.
                    \item Creación de la vista en Unity que se muestra al identificar una película mediante Realidad Aumentada, además de su funcionalidad y comunicación 
                    con Android. Ejemplo: mostrar tráiler, puntuación, información...
                    \item Creación de la vista en Unity que se muestra al identificar a un usuario no identificado como amigo. Así como la funcionalidad en Realidad Aumentada 
                    y Android para agregar a dicho usuario como amigo.
                    \item Creación de la vista en Unity que se muestra al identificar a un amigo en donde se muestran los tres planes recomendados y la información visual de cuánto 
                    se prevé que les guste ese plan a los usuarios dentro del plan.
                    \item Implementación en Android de la mayoría de las funciones que reciben información de Unity, realizan alguna petición al servidor y devuelven dicha información a Unity.
                    \item Creación en Android de métodos de comunicación con el servidor para acciones como añadir amigos, borrar plan, obtener amistad, obtener planes...
                    \item Creación en el servidor de la primera versión de usuarios y su relación con películas.
                \end{itemize}
            \subsection{Memoria}
            \label{makereference7.3.3}
                \begin{itemize}
                    \item Revisión de la memoria.
                    \item Aportaciones en distintas partes como herramientas usadas, contribuciones, dificultades encontradas.
                    \item Redacción de parte de las interfaces de usuario que se muestran en Unity y en Android.
                \end{itemize}
        
\section{Zihao Hong}
\label{makereference7.4}
    \subsection{Estado del arte}
    \label{makereference7.4.1}
        \begin{itemize}
            \item Estudio de la implementación de un prototipo que usara Vuforia y Android, además de las dependencias que necesitaba. Investigación del pipeline 
            que usan para renderizar objetos en el espacio, la librería de dibujo en 3D que se usa internamente OpenGL, 
            incluso aprendizaje de lenguajes de bajo nivel (GLSL) que se usa en las tarjetas gráficas. Para finalmente 
            decidir su viabilidad para incorporar al proyecto
            \item Investigación sobre los métodos de recomendación, mediante búsquedas de recursos en la web, prueba de 
            ejemplos dados, realización de simulaciones contra la base de datos visualizando los resultados y establecimiento 
            del método final del sistema.
            \item Estudio del funcionamiento del Cloud que ofrece Vuforia para el reconocimiento de imágenes. Simulaciones de ejemplos contra el Cloud y pruebas de servicios REST que ofrecen para 
            en un futuro la automatización de subidas de imágenes al Cloud.
            \item Investigación de la seguridad que necesita tener el servidor para restringir accesos.
        \end{itemize}
    \subsection{Diseño de la aplicación}
    \label{makereference7.4.2}
        \begin{itemize}
            \item Diseño de los primeros bocetos para los casos de usos y escenarios, proponiendo situaciones que se 
            podían dar en las que usar la aplicación y sus respectivas soluciones.
            \item Diseño de las interfaces de usuario de los amigos y en cómo se hacen peticiones de amistad, cómo responden a estas peticiones y el
            borrado lógico o físico de las amistades.
            \item Diseño de las interfaces de sesión de un usuario en la aplicación.
            \item Diseño inicial del esquema de relación de la base de datos.
        \end{itemize}
    \subsection{Implementación}
    \label{makereference7.4.3}
        \begin{itemize}
            \item Implementación de distintos ejemplos de proyectos Unity sobre realidad aumentada
            con Vuforia, verificando los distintos requisitos
            que necesitábamos para el proyecto. 
            \item Implementación de las primeras interfaces del proyecto de Unity con Vuforia generando dos 
            paneles al lado de la imagen del póster de la película para representar la información de la misma.
            \item Creación de prototipo con Vuforia en Android, utilizando los métodos que ofrecían para reconocer imágenes,
            e implementación con OpenGL para dibujar figuras sobre objetos mediante un lenguaje a bajo nivel (GLSL).
            \item Intento de escribir texto en el proyecto de Vuforia, el cual se consiguió mediante uso de mapa de bits, pero se rechaza 
            finalmente pues se trata de una implementación muy complicada.
            \item Creación de prototipo de la arquitectura de Microservicios, para posterior 
            incorporación en el proyecto. Utilizando un gateway llamado Zuul y un cloud de registros llamado Eureka, de los cuales 
            los demás servicios se registran y zuul los organiza.
            \item Evaluación del uso del Docker ofrecido por la universidad, creación de 
            scripts para imágenes de Docker y su posterior despliegue como contenedor en local, preparación de las distintas herramientas 
            necesarias para el despliegue del servidor en un entorno Linux, preparación de bases de datos dentro del contenedor, creación 
            de distintos perfiles para el despliegue en el proyecto del servidor y final implementación de scripts usados para activar el servicio 
            del servidor en Docker.
            \item Implementación inicial del servidor para su conexión con la base de datos junto con sus entidades más fundamentales.
            \item Implementación de rutas en el servidor, como consecuencia también sus servicios de aplicación, repositorios, etc.
            \item Creación de la base de la de arquitectura del proyecto de Android para establecer el flujo de actividades inicial en el proyecto de Android.
            \item Implementación de un Viewpager para integrar tres interfaces en una actividad para que el usuario pueda moverse entre ellas. Se establece como la principal 
            interfaz al entrar en la aplicación.
            \item Implementación de actividades de sesión y sus correspondientes controles. Se ha conseguido que la sesión del usuario siga activa incluso cuando se cierra 
            la aplicación, solo se termina cuando el usuario decide cerrar su sesión.
            \item Implementación de un panel de menú a la izquierda de la aplicación para darle más funcionalidad a la aplicación.
            \item Establecimiento de la clase RecyclerView para mostrar listas de objetos en la aplicación, ya que es la más usada en Android, la clase ListView está obsoleta.
            \item Creación de algunas vistas propias con características especiales para satisfacer las necesidades 
            que tiene la aplicación para mostrar algunas cosas.
            \item Establecimiento de la conexión del proyecto de Android con el proyecto de Unity mediante una actividad llamada UnityPlayerActivity, la cual recibe, ejecuta y retorna datos
            desde Unity. Para ello se ha usado el patrón de diseño de comando para abstraer dependencias entre Android y Unity.
            \item Diseño de una clase genérica para hacer peticiones contra el servidor y así facilitar su uso. Las respuestas se devuelven 
            en forma de callback ya que estas son asíncronas.
            \item Establecimiento de algunos efectos cuando se haga una transición de actividades dentro de la aplicación.
            \item Implementación en el servidor del sistema de recomendación usando la 
            librería de Mahout, que ofrece un método para la conexión con la base de datos 
            PostgreSQL sin tener que leer los datos de un fichero csv.
            \item Revisor de código en todos los proyectos, principalmente en Android y en el servidor para la generalización del estilo de código y correcciones 
            en cuanto se identifiquen errores.
            \item Establecimiento del uso de patrones de diseño, como comandos, singleton, 
            constructor, etc. Los cuales se encuentran en la implementación de Android. 
        \end{itemize}
    \subsection{Memoria}
    \label{makereference7.4.3}
        \begin{itemize}
            \item Explicación de técnicas de recomendación en el \autoref{makereference2.3}.
            \item Explicación del prototipo de Vuforia y Android en el apartado \autoref{makereference4.1.3}.
            \item Redacción de la parte cliente de la aplicación en el apartado \autoref{makereference4.4}.
            \item Explicación de sistemas de recomendación en el apartado \autoref{makereference4.3.2}.
            \item Redacción de la guía de despliegue del servidor en Docker en el \autoref{app:docker}.
        \end{itemize}
