% +--------------------------------------------------------------------+
% | Sample Chapter 7
% +--------------------------------------------------------------------+

\cleardoublepage

% +--------------------------------------------------------------------+
% | Replace "This is Chapter 7" below with the title of your chapter.
% | LaTeX will automatically number the chapters.
% +--------------------------------------------------------------------+

\chapter{Contribución al proyecto}
\label{makereference7}
En este capítulo especificaremos las aportaciones de cada integrante del grupo,
 las cuales se ordenaran por las fases del proyecto.

\section{Diego Acuña Berger}
\label{makereference7.1}
    \subsection{Estado del arte}
    \label{makereference7.1.1}
        \begin{itemize}
            \item
        \end{itemize}
    \subsection{Diseño de la aplicación}
    \label{makereference7.1.2}
        \begin{itemize}
            \item Diseños de vistas de la aplicación con \textbf{MockFlow}.
            \item Diseño de las interfaces de usuario de los planes y películas
            guardadas.
        \end{itemize}
    \subsection{Implementación}
    \label{makereference7.1.3}
        \begin{itemize}
            \item Estudio y creación del prototipo de servidor en \textbf{Spring}.
            \item Obtención de imágenes mediante \textbf{Picasso} a la aplicación.
            \item Funcionalidad para guardar películas mediante realidad aumentada.
            \item Disposición y diseño de botones que aparecen tras reconocer una
            película con realidad aumentada.
        \end{itemize}
    \subsection{Memoria}
    \label{makereference7.1.3}
        \begin{itemize}
            \item Redacción del resumen.
            \item Redacción de las distintas interfaces de usuario de la aplicación.
            \item Revisión de calidad de la memoria.
        \end{itemize}

\section{Daniel Calle Sánchez}
\label{makereference7.2}
    \subsection{Estado del arte}
    \label{makereference7.2.1}
        \begin{itemize}
            \item Estudio de la tecnologías de Realidad Aumentada, Viro React y
            Expo AR.
        \end{itemize}
    \subsection{Diseño de la aplicación}
    \label{makereference7.2.2}
        \begin{itemize}
            \item Identificación de requisitos funcionales de la aplicación.
            \item Diseños de vistas de la aplicación con MockFlow.
        \end{itemize}
    \subsection{Implementación}
    \label{makereference7.2.3}
        \begin{itemize}
            \item Creación de prototipo con ViroReact y ARCore.
            \item Diseño de la arquitectura.
            \item Configuración de Heroku.
            \item Implementación de servicios en Spring.
            \item Reconocimiento de usuarios con Vuforia en Unity.
            \item Desarrollo de vistas para planes en Android.
            \item Búsqueda y configuración de herramientas que mejoren el
             rendimiento del equipo.
        \end{itemize}
    \subsection{Memoria}
    \label{makereference7.2.3}
        \begin{itemize}
            \item Redacción de capítulo 5 Conclusiónes.
            \item Redacción de capítulo 6 Trabajo futuro.
        \end{itemize}

\section{Carlos Gómez Cereceda}
\label{makereference7.3}
    \subsection{Estado del arte}
    \label{makereference7.3.1}
        \begin{itemize}
            \item
        \end{itemize}
    \subsection{Diseño de la aplicación}
    \label{makereference7.3.2}
        \begin{itemize}
            \item
        \end{itemize}
    \subsection{Implementación}
    \label{makereference7.3.3}
        \begin{itemize}  
            \item Creación de prototipo con ARCore en Android.
            \item Creación de prototipo con Vuforia en Unity.
            \item Creación de prototipo con Vuforia en Unity exportado a
             Android.
            \item Configuración del proyecto para poder tener la parte de
             Vuforia separada de la de Android.
            \item Comunicación entre la parte de Vuforia en Unity y la de
             Android.
            \item Configuración del proyecto para que se pueda comunicar con el
             Cloud de Vuforia.
            \item Creación de las primeras vistas prototipo del proyecto, tales
             como: Mostrar vídeos en \textbf{Realidad Aumentada}, botones y
             películas.
            \item Comunicación en Vuforia con el servidor.
            \item Creación de la vista en Unity que se muestra al identificar
             una película mediante \textbf{Realidad Aumentada}, además de su
             funcionalidad y comunicación con \textbf{Android}. Ejemplo: mostrar
             tráiler, puntuación, información…
            \item Creación de la vista en Unity que se muestra al identificar a
             un usuario no identificado como amigo. Así como la funcionalidad en
             Realidad Aumentada y Android para agregar a dicho usuario como
             amigo.
            \item Creación de la vista en Unity que se muestra al identificar a
             un amigo en donde se muestran los tres planes recomendados y la
             información visual de cuánto se prevé que les guste ese plan a los
             usuarios dentro del plan.
            \item Creación en Android de métodos de comunicación con el servidor
             para acciones como añadir amigos, borrar plan, añadir amigo,
             obtener amistad…
            \item Creación en el servidor de la primera versión de usuarios y su
             relación con películas.
        \end{itemize}
    \subsection{Memoria}
    \label{makereference7.3.3}
        \begin{itemize}
            \item
        \end{itemize}

\section{Zihao Hong}
\label{makereference7.4}
    \subsection{Estado del arte}
    \label{makereference7.4.1}
        \begin{itemize}
            \item
        \end{itemize}
    \subsection{Diseño de la aplicación}
    \label{makereference7.4.2}
        \begin{itemize}
            \item
        \end{itemize}
    \subsection{Implementación}
    \label{makereference7.4.3}
        \begin{itemize}  
            \item Creación de prototipo con Vuforia en Android.
            \item Creación de prototipo de la arquitectura de Microservicios.
            \item Evaluación y pruebas para establecer el proyecto de Unity.
            \item Evaluación del uso del Docker ofrecido por la universidad.
            \item Diseño de arquitectura del proyecto de android y de las
             vistas.
            \item Configuración del proyecto del servidor mediante maven y
             perfiles para el despliegue en un entorno Linux.
            \item Estudio de sistemas de recomendación y su implementación en el
            servidor.
            \item Revisor de código.
            \item Establecimiento del uso de patrones de diseño. 
        \end{itemize}
    \subsection{Memoria}
    \label{makereference7.4.3}
        \begin{itemize}
            \item
        \end{itemize}
